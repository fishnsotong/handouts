\section{Conductors and Capacitance}
\subsection*{Conductors}
(1) $\mathbf{E} = 0$ inside a conductor

(2) $\rho = 0$ inside a conductor

(3) $\phi = \phi_k$ at all points

(4) $\mathbf{E}$ is perpendicular to the surface at any point on it, $E = \sigma / \epsilon_0$.

(5) $Q_k = \int_S \sigma \differential a = \epsilon_0 \int_S \mathbf{E} \cdot \differential \mathrm{a}$.

\subsection*{Capacitance}

The \textit{capacitance} $C$ is defined by $Q=C\phi$, gives a measure of how much charge a conductor can hold
for a given potential.

For a sphere and parallel-plate capacitor, their capacitances are:
\begin{equation*}
    C_{\textrm{sphere}} = 4 \pi \epsilon_0 r \quad C_{\textrm{plates}} \frac{\epsilon_0 A}{L}
\end{equation*}

The energy in a capacitor is
\begin{equation*}
    U = \frac{1}{2}C\phi^2 = \frac{Q^2}{2C} = \frac{1}{2}Q\phi
\end{equation*}
Between two parallel conducting plates, $\Delta \phi = Ed$. The energy density (energy stored per unit volume) is:
\begin{equation*}
    u_d = \frac{1}{2}\epsilon_0 E^2.
\end{equation*}

The electric field is where the potential energy put into a capacitor is stored.

\section{Dipoles and Dielectrics}
A \textit{dipole} has two charges $\pm q$, a distance $l$ apart. The \textit{dipole moment} is $p = ql$. At 
large distances, the potential and field due to a dipole are
\begin{equation*}
    \begin{aligned}
        \phi(r,\theta) &= \frac{p \cos \theta}{4 \pi \epsilon_0 r^2} \\
        \mathbf{E}(r,\theta) &= \frac{p}{4 \pi \epsilon_0 r^3}(2 \cos \theta \mathbf{\hat{r}} + \sin \theta \pmb{\hat{\theta}})
    \end{aligned}
\end{equation*}

\subsection*{Dipoles in a field}
Torque experienced by a dipole:
\begin{equation*}
    \mathbf{\tau} = \mathbf{p} \times \mathbf{E}
\end{equation*}
If the angle between the field and the dipole is $\theta$, $|\tau| = pE\sin \theta$.

Potential energy of a dipole in a field is work needed to put the dipole in position
\begin{equation*}
    U = - \mathbf{p} \cdot \mathbf{E}
\end{equation*}
Polarisability of a molecule is $\mathbf{p} = \alpha \mathbf{E}$.

Polarisation is the dipole moment per unit volume of the material $\mathbf{P} = \sum \mathbf{p} / V$.

\subsection*{Dielectrics}
Where $C_0$ is capacitance in vacuum, the dielectric constant is $K = C/C_0$. $\mathbf{E}$ within
the capacitor is reduced because of the dielectric.