
\section{Electrostatics}

Electric charge, which can be positive or negative, is both \textit{conserved} and \textit{quantised}.

\subsection*{Coulomb's law}
The force between two charges is
\begin{equation*}
  \mathbf{F} = \frac{1}{4\pi\epsilon_0}\frac{q_1q_2}{r_{21}}
\end{equation*}
\textit{Superposition.} The force with which two charges interact is not changed by the presence of a third charge. 
Integrating this force, the \textit{potential energy} (work done in moving the charges from infinity) is
\begin{equation*}
  U = \frac{1}{4\pi\epsilon_0}\frac{q_1q_2\hat{\mathbf{r}_{21}}}{r^2_{21}}
\end{equation*}
For a system of many charges,
\begin{equation*}
  U = \frac{1}{2}\sum_{j=1}^{N}\sum_{k \neq j}^{} \frac{1}{4\pi\epsilon_0}\frac{q_j q_k}{r_jk}
\end{equation*}
\subsection*{Electric field}
The \textit{electric field} due to a charge distribution is (units: $\mathrm{NC}^{-1}$):
\begin{equation*}
  \mathbf{E} = \frac{1}{4\pi\epsilon_0} \sum_{j=1}^{N} \frac{q_j\hat{\mathbf{r}}_j}{r^2_j}
\end{equation*}
The electric field attaches to every point in the system a \textit{local property}: the force per unit charge that test charge $q_0$ will experience at any point.\\
The force on a test charge $q$ is $\mathbf{F} = q \mathbf{E}$. \\
\subsection*{Flux}
Flux is defined as the surface integral of the perpendicular component of a vector field over a surface. 
\begin{equation*}
  \Phi = \int_S \mathbf{E} \cdot \differential \mathbf{a}
\end{equation*}
Mathematically, $\Phi_E$ is the integral of the normal component of the electric field over a given area. 
Conceptually, $\Phi_E$ is the `amount of field lines' passing through a given area: $\mathbf{E} \cdot \mathbf{A} = |E||A|\cos \theta$.

\subsection*{Gauss' law}
Flux of $\mathbf{E}$ through any closed surface equals $1/\epsilon_0$ times total charge enclosed
by the surface.
\begin{equation*}
  \int_S \mathbf{E} \cdot \differential \mathbf{a} = \frac{1}{\epsilon_0} \int p \differential v = \frac{1}{\epsilon_0} \sum_{i} q_i
\end{equation*}
Net electric flux $\Phi_E$ out is proportional to charge enclosed.

For a sphere:
\begin{equation*}
  E = \frac{Q}{4\pi\epsilon_0r^2}
\end{equation*}


For a line (what is $\lambda$?):
\begin{equation*}
  E = \frac{\lambda}{2\pi\epsilon_0r}
\end{equation*}
For a sheet:
\begin{equation*}
  E = \frac{\sigma}{2\epsilon_0}
\end{equation*}

Gauss' law is always valid, but it is only useful in calculating $\mathbf{E}$ in cases with enough symmetry. 

$\mathbf{E}$ is also known as the \textit{flux density.}

\textbf{Corollaries for Gauss' law.}

(1) Flux through any surface surrounding the same charges is the same.

(2) Charges outside the surface do not affect the flux through the surface.

(3) The enclosed charge can be inferred from the field at a closed surface.

(4) The electric field \textit{outside} a \textbf{spherical} charge distribution of total charge $Q$ is the same
as for a point charge $Q$ at the centre.