\section{Magnetic Fields}
Charged particles in a \textit{magnetic field} $\mathbf{B}$ experience a magnetic force when moving:
\begin{equation*}
    \begin{aligned}
        \mathbf{F}\ &= q\mathbf{v} \times \mathbf{B} \\
        F = Bqv\sin &\theta \quad B = \frac{F}{qv\sin \theta}
    \end{aligned}
\end{equation*}
combining with $\mathbf{F} = q \mathbf{E}$ for stationary charges, we get the \textit{Lorentz force}
on a charged particle in an electromagnetic field
\begin{equation*}
    \mathbf{F} = q \mathbf{E} + q\mathbf{v} \times \mathbf{B}.
\end{equation*}
\subsection*{Gauss' law for magnetic fields}
\textit{Magnetic flux} $\Phi_B$ is analogous to $\Phi_E$.

\begin{equation*}
    \Phi = \int_S \mathbf{B} \cdot \differential \mathbf{a}
\end{equation*}
Gauss' law also works for $\mathbf{B}$: The total magnetic flux passing through any closed surface is zero.
\begin{equation*}
    \int_S \mathbf{B} \cdot \differential \mathbf{a} = 0
\end{equation*}
This means that $\divergence B = 0$. There are no `sources' of magnetic field: no monopoles, and magnetic field lines have no endings.

\subsection*{Ampere's law}
For \textit{steady currents}, $\mathbf{B}$ only depends on $I$, the rate of charge transport:
\begin{equation*}
    \int \mathbf{B} \cdot \differential \mathbf{s} = \mu_0 I
\end{equation*}
Useful for determining $\mathbf{B}$ for cases with a high degree of symmetry. \\
\textbf{Field from a long wire}: $\mathbf{B}$ points in the tangential direction and has magnitude
\begin{equation*}
    B = \frac{\mu_0 I}{2\pi R}
\end{equation*}
Field inside thick circular wire (radius $a$):
\begin{equation*}
    B = \frac{\mu_0 Ir}{2\pi a^2}
\end{equation*}
Field at the centre of a current loop:
\begin{equation*}
    B = \frac{\mu_0 I}{2R}
\end{equation*}
Field due to infinitely long solenoid:
\begin{equation*}
    B = \mu_0nI
\end{equation*}
where $n$ is the number of turns.

\subsection*{Biot-Savart law}
The contribution to $\mathbf{B}$ from a piece of a wire carrying current $I$ is given by:
\begin{equation*}
    \differential \mathbf{B} = \frac{\mu_0 I}{4 \pi} \frac{\differential \mathbf{l} \times \hat{\mathbf{r}}}{r^2} = \frac{\mu_0 I}{4 \pi} \frac{\differential \mathbf{l} \times \mathbf{r}}{r^3}
\end{equation*}
A current element $\differential \mathbf{l}$ must be part of a complete circuit to conserve charge.

\textbf{Field from a finite wire}: For a finite wire of length $a+b$:
\begin{equation*}
    B = \frac{\mu_0 I}{4 \pi d}(\cos \beta - \cos \alpha)
\end{equation*}
Field at centroid of triangular coil:
\begin{equation*}
    B = 3\sqrt{3}\frac{\mu_0 I}{4 \pi d}
\end{equation*}
\subsection*{Particle motion in E and B fields}
In a \textit{uniform magnetic field}, charged particles undergo circular orbits, moving in a spiral around $\mathbf{B}$ field lines.
\begin{equation*}
    \nu = \frac{1}{T} = \frac{v}{2\pi R} = \frac{qB}{2 \pi m}
\end{equation*}

In crossed $\mathbf{E}$ and $\mathbf{B}$ fields, we can make \textit{velocity selectors}, which can be coupled to \textit{mass spectrometers}. If $qvB = qE$, $v = E/B$.

\subsection*{Effect of magnetic field on wires}
Current is $I = nAve$.\\
Current density is $J = I / A$, $J = nqe$

\textbf{Motors}: $I$ through a wire loop in a $\mathbf{B}$ field can lead to a torque, generating motion
\begin{equation*}
    \mathbf{\tau} = I(\mathbf{A} \times \mathbf{B})
\end{equation*}
\textbf{Force between two parallel wires}: Like currents attract, unlike currents repel.
\begin{equation*}
    \frac{F}{l} = \frac{\mu_0 I_1 I_2}{2 \pi r}
\end{equation*}
\section{Electromagnetic Induction} 

\subsection*{Faraday's law}
\textit{Faraday's law of induction}
\begin{equation*}
    \epsilon = - \dv{\Phi_B}{t}
\end{equation*}
The loop can be moving, or the source of the $\mathbf{B}$ field can be moving, or otherwise.
\subsection*{Lenz's law}
The direction of the induced emf is such that the induced current creates a $\mathbf{B}$ field that opposes the change in flux.

\textbf{Generator}: Consider a rotating loop with vector area $\mathbf{A}$ in uniform constant $\mathbf{B}$. Flux through the loop is $\Phi = \mathbf{B} \cdot \mathbf{A} = BA \cos \omega t$.
\begin{equation*}
    \epsilon = - \dv{\Phi_B}{t} = BA \omega \sin \omega t
\end{equation*}

