\section{Maxwell's Equations}

\subsection*{Integral form of Gauss' law}
\begin{equation*}
    \oint_S \mathbf{E} \cdot \mathbf{\hat{n}} \differential a = \frac{q_\textrm{enc}}{\epsilon_0}
  \end{equation*}
Electric charge produces an electric field, and the flux of that field passing through any closed surface is 
proportional to total charge contained within that surface.

\textbf{Types of problems} \\
(1) Given information about a distribution of electric charge, you can find the electric flux through a surface enclosing that charge.

(2) Given information about the electric flux through a closed surface, you can find the total electric charge enclosed by that surface.

  \subsection*{Differential form of Gauss' law}
  \begin{equation*}
    \divergence \mathbf{E} = \frac{\rho}{\epsilon_0}
  \end{equation*}
  The electric field produced by electric charge diverges from positive charge and converges upon negative charge.

\subsection*{Integral form of Gauss' law for magnetism}
\begin{equation*}
    \oint_S \mathbf{B} \cdot \mathbf{\hat{n}} \differential a = 0
  \end{equation*}
  The total magnetic flux passing through any closed surface is zero.

\subsection*{Gauss' law for magnetism: differential}
\begin{equation*}
    \div \mathbf{B} = 0
\end{equation*}
The divergence of the magnetic field at any point is zero.

\subsection*{Faraday's law: integral form}
\begin{equation*}
    \oint_C \mathbf{E} \cdot \differential \mathbf{l} = -\dv{t}\int_S \mathbf{B} \cdot \mathbf{\hat{n}} \differential a
\end{equation*}
Changing magnetic flux through a surface induces an emf in any boundary path of that surface, and a changing magnetic field induces a circulating electric field.

\textbf{Types of problems} \\
(1) Given information about the changing magnetic flux, find the induced emf.

(2) Given the induced emf on a specified path, determine the rate of change of the magnetic field magnitude or direction or the area bounded by the path.

\subsection*{Faraday's law: differential form}
\begin{equation*}
    \curl \mathbf{E} = - \pdv{\mathbf{B}}{t}
\end{equation*}
A circulating electric field is produced by a magnetic field that changes with time.

\subsection*{Ampere-Maxwell law: integral form}
\begin{equation*}
    \oint_C \mathbf{B} \cdot \differential \mathbf{l} = \mu_0 \left(I_{\textrm{enc}} + \epsilon_0 \dv{t}\int_S \mathbf{E} \cdot \mathbf{\hat{n}} \differential a\right)
\end{equation*}
An electric current or a changing electric flux through a surface produces a circulating magnetic field around any path that bounds that surface.

\subsection*{Ampere-Maxwell law: differential form}
\begin{equation*}
    \curl B = \mu_0 \left(\mathbf{J} + \epsilon_0 \pdv{\mathbf{E}}{t}\right)
\end{equation*}
A circulating magnetic field is produced by an electric current and by an electric field that changes with time.

\section{Electromagnetic Waves}
The power density (energy per unit area per unit time) of an electromagnetic wave is the \textit{Poynting vector}:
\begin{equation*}
    \mathbf{S} = \frac{1}{\mu_0}\mathbf{E} \times \mathbf{B}
\end{equation*}
The waves are transverse, perpendicular to the direction of travel.

Transverse oscillations in $\mathbf{E}$ are perpendicular to transverse oscillations in $\mathbf{B}$.

Total energy stored per unit volume in an electromagnetic field is:
\begin{equation*}
    e_{d} = \frac{1}{2}\epsilon_0 E^2 + \frac{1}{2}\frac{B^2}{\mu_0}
\end{equation*}
The energy carried by the wave is shared equally between \textbf{E} and \textbf{B} fields.

\subsection*{Polarisation}

\subsection*{Circular polarisation}
Where polarisations do not have the same phase, but there is a phase difference of $\phi = 90^{\circ}$.