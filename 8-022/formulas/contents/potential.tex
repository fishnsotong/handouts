\section{Electric Potential}
The electric potential $\phi$ due to a charge distribution is potential energy per unit charge ($U = q\phi$):
\begin{equation*}
    \phi = \sum_{i} \frac{q_i}{4 \pi \epsilon_0}
\end{equation*}
The energy required to assemble a group of charges from infinity is 
\begin{equation*}
    U = \frac{1}{2} \int \rho \phi \differential v = \frac{1}{2}\sum_{j=1}^{N}q_j \phi_j
\end{equation*}
The \textit{gradient} of a scalar function $\nabla f$ gives the direction in which $f$ has the largest
rate of increase.
\begin{equation*}
    \mathbf{E} = - \grad \phi
\end{equation*}
Lines of electric field are perpendicular to surfaces of constant potential $\phi$.

\subsection*{Potential difference}
The \textit{electric potential difference} is:
\begin{equation*}
    \phi_2 - \phi_1 \quad = \int_{P_1}^{P_2} \grad \phi \cdot \differential \mathbf{s}
\end{equation*}
As $\mathbf{E}$ is conservative, potential difference is path independent.

The electric \textit{potential energy} difference is:
\begin{equation*}
    U_{21} = q(\phi_2 - \phi_1)
\end{equation*}