\documentclass[a4paper]{tufte-handout}

\title{24. Maxwell's Equations\thanks{Wayne~W.~Z. Yeo}}

\author[MIT 8.022]{\textnormal{MIT 8.022} Electricity \& Magnetism\thanks{Course Instructors: Gabriella Sciolla, Walter Lewin}}

%\date{28 March 2010} % without \date command, current date is supplied

%\geometry{showframe} % display margins for debugging page layout

\usepackage{graphicx} % allow embedded images
  \setkeys{Gin}{width=\linewidth,totalheight=\textheight,keepaspectratio}
  \graphicspath{{graphics/}} % set of paths to search for images
\usepackage{amsmath,amsthm}  % extended mathematics
\usepackage{physics}
\usepackage[version=4]{mhchem}
\usepackage{booktabs} % book-quality tables
\usepackage{units}    % non-stacked fractions and better unit spacing
\usepackage{multicol} % multiple column layout facilities
\usepackage{lipsum}   % filler text
\usepackage{fancyvrb} % extended verbatim environments
  \fvset{fontsize=\normalsize}% default font size for fancy-verbatim environments

% Standardize command font styles and environments
\newcommand{\doccmd}[1]{\texttt{\textbackslash#1}}% command name -- adds backslash automatically
\newcommand{\docopt}[1]{\ensuremath{\langle}\textrm{\textit{#1}}\ensuremath{\rangle}}% optional command argument
\newcommand{\docarg}[1]{\textrm{\textit{#1}}}% (required) command argument
\newcommand{\docenv}[1]{\textsf{#1}}% environment name
\newcommand{\docpkg}[1]{\texttt{#1}}% package name
\newcommand{\doccls}[1]{\texttt{#1}}% document class name
\newcommand{\docclsopt}[1]{\texttt{#1}}% document class option name

\setlength\fboxsep{7px}

\newenvironment{docspec}{\begin{quote}\noindent}{\end{quote}}% command specification environment

\newtheorem{theorem}{Theorem}
\newtheorem{corollary}{Corollary}
\newenvironment{justification} {\begin{proof}[Justification]} {\end{proof}}

\theoremstyle{definition}
\newtheorem{definition}{Definition}
\newtheorem{example}{Example}


\begin{document}

\maketitle% this prints the handout title, author, and date

\begin{abstract}
\noindent
In this lecture, we finally put everything together and get the full set of "field equations" which describe electricity and magnetism -- 
\textbf{Maxwell's equations}, fully developed by James Clerk Maxwell in 1873.

\end{abstract}

%\printclassoptions

\section{Introduction}
\textbf{Books.} This topic has been given a thorough discussion in Chapter 9 of Purcell \& Morin\cite{purcell2013electricity} -- I will refer to these texts from time to time.

\section*{Maxwell's equations in integral form}
To wrap this section up, let's write Maxwell's equations in the fashion in which they are most useful: in integral form. These
are easily derived by plugging the differential forms into integrals and invoking various vector theorems

\begin{theorem}[Integral form of Gauss' law]
  The electric flux through a closed surface $S$ is the product of $4\pi$ and the charge enclosed by $S$.
  \begin{equation}
    \Phi_E (S) = \oint_S \vec{E} \cdot \differential \vec{a} = 4 \pi Q_{\mathrm{encl}}
  \end{equation}
\end{theorem}

\begin{theorem}
  The magnetic flux through a closed surface $S$ is zero.\marginnote{Follows quite naturally from our observation that there are no magnetic point charges!}
  \begin{equation}
    \Phi_B (S) = \oint_S \vec{B} \cdot \differential \vec{a} = 0
  \end{equation}
\end{theorem}

\begin{theorem}[Integral form of Faraday's law]
  The emf induced around a closed contour $C$ equals the product of $-1/c$ and the rate of change of magnetic flux
  through the surface $S$ bounded by $C$.
  \begin{equation}
    \epsilon = \oint_C \vec{E} \cdot \differential \vec{s} = -\frac{1}{c}\pdv*{f}{x}
  \end{equation}
\end{theorem}

\begin{theorem}
  The line integral of the $\vec{\mathbf{B}}$ field around a closed contour $C$ is the product of $4 \pi / c$ and the sum of the
  total current (real and displacement) passing through that contour.
  \begin{equation}
    \oint_C \vec{B} \cdot \differential \vec{s} = \frac{4 \pi}{c} (I + I_d)
  \end{equation}
  where
  \begin{align*}
    I &= \int_S \vec{J} \cdot \differential a \\
    I_d &= \int_S \vec{J}_d \cdot \differential a = \frac{1}{4 \pi}\pdv{\Phi_E(S)}{t}
  \end{align*}
\end{theorem}

Once we have written down the full set of Maxwell's equations, we discover that in vacuum they lead to wave solutions
with a set of specific properties.

\bibliography{lec24}
\bibliographystyle{plainnat}
\end{document}
