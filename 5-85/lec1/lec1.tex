\documentclass[a4paper]{tufte-handout}

\title{1.Introduction \thanks{Wayne~W.~Z. Yeo}}

\author[MIT 5.85]{\textnormal{MIT 5.85} Molecular Reaction Dynamics\thanks{Course Instructors: Robert Field, Niels E. Henriksen \& Flemming Y. Hansen}}

%\date{28 March 2010} % without \date command, current date is supplied

%\geometry{showframe} % display margins for debugging page layout

\usepackage{graphicx} % allow embedded images
  \setkeys{Gin}{width=\linewidth,totalheight=\textheight,keepaspectratio}
  \graphicspath{{graphics/}} % set of paths to search for images
\usepackage{amsmath,amsthm}  % extended mathematics
\usepackage{physics,siunitx}
\usepackage[version=4]{mhchem} \usepackage{chemmacros}
\usepackage{booktabs} % book-quality tables
\usepackage{units}    % non-stacked fractions and better unit spacing
\usepackage{multicol} % multiple column layout facilities
\usepackage{lipsum}   % filler text
\usepackage{fancyvrb} % extended verbatim environments
  \fvset{fontsize=\normalsize}% default font size for fancy-verbatim environments

% Standardize command font styles and environments
\newcommand{\doccmd}[1]{\texttt{\textbackslash#1}}% command name -- adds backslash automatically
\newcommand{\docopt}[1]{\ensuremath{\langle}\textrm{\textit{#1}}\ensuremath{\rangle}}% optional command argument
\newcommand{\docarg}[1]{\textrm{\textit{#1}}}% (required) command argument
\newcommand{\docenv}[1]{\textsf{#1}}% environment name
\newcommand{\docpkg}[1]{\texttt{#1}}% package name
\newcommand{\doccls}[1]{\texttt{#1}}% document class name
\newcommand{\docclsopt}[1]{\texttt{#1}}% document class option name
\newenvironment{docspec}{\begin{quote}\noindent}{\end{quote}}% command specification environment

\newtheorem{theorem}{Theorem}
\newtheorem{corollary}{Corollary}
\newtheorem{definition}{Definition}
\newenvironment{justification} {\begin{proof}[Justification]} {\end{proof}}

\begin{document}

\maketitle % this prints the handout title, author, and date

\begin{abstract}
\noindent
Transition metals are elements whose atoms or cations possess partially filled d subshells, which result
in them being stable at various oxidation states and forming coordination compounds with a wide diversity of
ligands.
\end{abstract}

%\printclassoptions

\newthought{What is the nature} of the forces at play in chemical reactivity? How can we 
see the invisible and measure the ephemeral? 

\section*{The simplest chemical reaction -- \textnormal{[\ce{H2 + H -> H + H2}]}}

Because this reaction is constrained to react in a linear geometry, it has a \textit{perfectly symmetrical}
potential energy. The transition state is in the centre of the curve of the 'half pipe'.

\section{Summary}

\begin{itemize}
  \item Alkyl-metal and aryl-metal complexes are thermodynamically stable, but kinetically labile.
  \item $\beta$-hydrogen elimination 
  \item These processes are essentially in a number of catalytic processes, such as alkene polymerisation, cross-couplings and hydroformylations.
  \item Also, they are key intermediates in \textbf{\ce{C-H} activation} processes. The transformation of unfunctionalised alkanes and arenes is one of the most active areas of research in the past twenty years \cite{guillemard2021late}.
\end{itemize}

\bibliography{lec1}
\bibliographystyle{plainnat}

\end{document}