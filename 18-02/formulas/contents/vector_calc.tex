\section{Vector Differential Calculus}

\subsection*{Gradient}
For a function $\phi(x, y, z)$,
\begin{equation*}
    \nabla \phi = \pdv{\phi}{x} \textbf{\textit{i}} + \pdv{\phi}{y} \textbf{\textit{j}} + \pdv{\phi}{z} \textbf{\textit{k}}
\end{equation*}

$\nabla \phi$ points in the direction of maximum increase of $\phi$. $|\nabla \phi|$ gives the slope (rate of change) along this direction.

\textbf{Directional derivatives.} Where $\textbf{\textit{t}}$ is a \textit{unit vector},
the following equation is the rate of change of $\phi$ with distance in the direction of $\textbf{\textit{t}}$
\begin{equation*}
    \textbf{\textit{t}} \cdot \nabla \phi = \dv{\phi}{s}
\end{equation*}

\textbf{Normal to a surface ($\phi = c$).} The \textit{unit normal vector} to that surface at any point
\begin{equation*}
    \textbf{\textit{n}} = \frac{\nabla \phi}{|\phi|}.
\end{equation*}
Given $\textbf{\textit{n}}$, we can find the tangent plane.
\subsection*{Divergence}
For a vector field $\mathbf{A}$ in $\mathbb{R}^3$
\begin{equation*}
    \nabla \cdot \mathbf{A} = \pdv{A_x}{x} + \smash{\pdv{A_y}{y}}+ \pdv{A_z}{z}
\end{equation*}

Per definition of dot product, for $\mathbb{R}^n$
\begin{equation*}
    \nabla \cdot \mathbf{A} = \sum_{i}^{n} \pdv{A_i}{x_i}
\end{equation*}

$\divergence \mathbf{A}$ is the net flux out per unit volume around a point in a vector field.

\subsection*{Curl}
For a vector field $\mathbf{A}$ in $\mathbb{R}^3$
\begin{equation*}
    \nabla \times \mathbf{A} = \mdet{\mathbf{i} & \mathbf{j} & \mathbf{k}\\ \partial_x & \partial_y & \partial_z \\ A_x & A_y & A_z}
\end{equation*}
$\nabla \times \mathbf{A}$ is a measure of how much vector $\mathbf{A}$ swirls/rotates around a particular point.
If $\nabla \times \mathbf{E} = 0$, $\mathbf{E}$ is a conservative vector field. \textit{What does it mean for a field to be conservative? See Gradient Theorem.}

\subsection*{Laplacian}
Take note that $\nabla^2 \phi = \divergence \grad \phi$ as we take the inner product of two gradient operators $\nabla \cdot \nabla = \nabla^2$.
\begin{equation*}
    \nabla^2 \phi =  \pdv[2]{\phi}{x} + \pdv[2]{\phi}{x} + \pdv[2]{\phi}{x}
\end{equation*}

$\nabla^2 \phi$ is a useful measure of convexity in higher dimensions.
\vspace{\baselineskip}

For useful identities regarding the above differential operators, see the formula booklet.

\section{Vector Integral Calculus}

\subsection*{Line Integrals}
Line integral over a \textbf{scalar field} $\phi(\mathbf{x})$ is
\begin{equation*}
    \int_C \phi \differential s = \int_{t_1}^{t_2} \phi(\mathbf{x}(t)) \left\lvert \dv{\mathbf{x}}{t} \right\rvert \differential t
\end{equation*}

When parameterisation is by arc length $s$ we can use a simpler form $\because |\differential x| = |\differential s|$.

\begin{equation*}
    \int_C \phi \differential s = \int_{t_1}^{t_2} \phi(\mathbf{x}(s)) \differential s
\end{equation*}

Line integral over a \textbf{vector field} $\mathbf{F}(\mathbf{x})$ is
\begin{equation*}
    \int_C \mathbf{F} \cdot \differential \mathbf{x} = \int_{t_1}^{t_2} \mathbf{F}(\mathbf{x}(t))\cdot \dv{\mathbf{x}}{t}\differential t
\end{equation*}

\textbf{Properties of line integrals.} If a curve is reversed in direction, the line integral changes sign.
\begin{equation*}
    \int_{- \Gamma} \mathbf{F} \cdot \differential \mathbf{x} = - \int_{\Gamma} \mathbf{F} \cdot \differential \mathbf{x}
\end{equation*}

If curves are added, the line integrals also add up

\begin{equation*}
    \int_{\Gamma_1 + \Gamma_2} \mathbf{F} \cdot \differential \mathbf{x} = \int_{\Gamma_1} \mathbf{F} \cdot \differential \mathbf{x} + \int_{\Gamma_2} \mathbf{F} \cdot \differential \mathbf{x}
\end{equation*}

\subsection*{Gradient Theorem}
Line integral of the gradient of a scalar field $\phi$ is given by the value of $\phi$ at the boundaries $1$ and $2$.

\begin{equation*}
    \phi_2 - \phi_1 = \int\limits_{\mathrm{curve}} \grad \phi \cdot \differential \mathbf{s}
\end{equation*}
\textbf{Corollaries.} \\
If a vector field is the gradient of a scalar field, its line integral is \textit{path independent}.
\begin{equation*}
    \oint \grad \phi \cdot \differential \bf{x} = 0
\end{equation*}
Since the start and end points are the same for a closed integral, $\phi_2 - \phi_1 = 0$.

\subsection*{Conservative Vector Fields}

 \textbf{Conservative forces.} Certain force fields, like the gravitational or electrostatic force can be written in the form $\mathbf{F} = -\grad \phi$.
 
 \textbf{Potential.} The scalar field $\phi(\mathbf{x})$ is known as the \textit{potential}.

 \textbf{Work.} Total work done by a force field $\mathbf{F}(\mathbf{x})$ when particle moves along curve $C$
 \begin{equation*}
    W = \int_{C} \bf{F} \cdot \differential \bf{x}
 \end{equation*}

By the \textit{gradient theorem},
 \begin{equation*}
    W = - \int_{C} \bf{\grad \phi} \cdot \differential \bf{x} = \phi(\mathbf{x}_{\mathrm{1}}) - \phi(\mathbf{x}_{\mathrm{2}})
 \end{equation*}

 Work done equal to the \textit{potential difference} between the endpoints; is independent of path taken by the object between them. 
 
 $$\oint \mathbf{F} \cdot \differential \mathbf{x} = 0$$
 
 for \textit{all} closed curves. No net work done around a closed path as there is no net change in potential energy.
 
 \textbf{Definition of conservative field.} \\
 (1) $\mathbf{F} = -\grad \phi$ \\
 (2) $\int_C \mathbf{F} \cdot \differential \mathbf{x}$ is path independent \\
 (3) $\oint \mathbf{F} \cdot \differential \mathbf{x} = 0$ \\
 (4) $\nabla \times \mathbf{F} = 0$

\subsection*{Surface Integrals}

A \textbf{plane surface} in 3D space has \textit{vector area} $\mathbf{S} = A \mathbf{n}$. The \textit{vector area element} is $\differential \mathbf{S} = \mathbf{n} \differential S$ as it touches the tangent plane at some point.

The \textbf{total vector area} of $S$ is:
\begin{equation*}
    \mathbf{S} = \int_{S} \differential \mathbf{S} = \int_{S} \mathbf{n} \differential S
\end{equation*}
A \textbf{curved surface} $S$ in 3D space can be defined by a vector-valued function with two parameters $\mathbf{X}(s,t)$. The vector area element is an infinitesimal parallelogram given by the following cross product:
\begin{equation*}
    \begin{aligned}
    \differential \mathbf{S} &= \left(\pdv{\mathbf{X}}{s} \differential s\right) \times \left(\pdv{\mathbf{X}}{t} \differential t\right) \\
    &= \left(\pdv{\mathbf{X}}{s} \times \pdv{\mathbf{X}}{t} \right) \differential s \differential t
    \end{aligned}
\end{equation*}
The \textbf{flux} of vector field $\mathbf{F}$ through surface $S$ given by $\phi = c$ is:
\begin{equation*}
    \int_{S} \mathbf{F} \cdot \differential \mathbf{S} = \int_{S} \mathbf{F} \cdot \mathbf{n} \differential S = \iint \mathbf{F} \cdot \frac{\grad \phi}{|\grad \phi|} \differential S
\end{equation*}

To evaluate such an integral, we generally need to find a parametrisation of the surface, and expressions for $\mathbf{n}$ and $\differential S$. We
then need to calculate a double integral over the surface.

\subsection*{Volume Integrals}
A volume integral where $T$ is a scalar function, $\differential \tau$ is an infinitesimal volume element is, in Cartesian coordinates:
\begin{equation*}
    \int_{V} T \differential \tau = \iiint T \differential x \differential y \differential z
\end{equation*}
You can do the integrals in any order. 

The infinitesimal volume element $\differential \tau$, in spherical polar coordinates:

$\differential \tau = \differential l_r \differential l_\theta \differential l_\phi = r^2 \sin \theta \differential r \differential \theta \differential \phi$

Volume integrals are simpler than line or surface integrals because $\mathrm{d}V$ is a scalar quantity.

\subsection*{Gauss' Theorem (Divergence Theorem)}
\begin{equation*}
    \int\limits_{\mathrm{surface}} \mathbf{F} \cdot \differential \mathbf{a} \quad = \int\limits_{\mathrm{volume}} \divergence \mathbf{F} \differential v
\end{equation*}
\subsection*{Stokes' Theorem (Curl Theorem)}
\begin{equation*}
    \int\limits_{\mathrm{curve}} \mathbf{A} \cdot \differential \mathbf{s} \quad = \int\limits_{\mathrm{surface}} \curl \mathbf{A} \cdot \differential \mathbf{a}
\end{equation*}
\subsection*{Note on Integral Theorems}
Generally, the \textit{integral} of a \textit{derivative} over a \textbf{region} is equal to the function's value at the \textbf{boundary}.
\begin{equation*}
    \begin{aligned}
    \phi_2 - \phi_1 \quad &= \int\limits_{\mathrm{curve}} \grad \phi \cdot \differential \mathbf{s} \\
    \int\limits_{\mathrm{surface}} \mathbf{F} \cdot \differential \mathbf{a} \quad &= \int\limits_{\mathrm{volume}} \divergence \mathbf{F} \differential v \\
    \int\limits_{\mathrm{curve}} \mathbf{A} \cdot \differential \mathbf{s} \quad &= \int\limits_{\mathrm{surface}} \curl \mathbf{A} \cdot \differential \mathbf{a}
\end{aligned}
\end{equation*}
\subsection*{Green's Theorem}
A special case of Gauss' Theorem applied to the vector field $\psi \grad \phi$.
\begin{equation*}
    \begin{aligned}
        \int_{S} \psi \nabla \phi \cdot \differential \mathbf{S} &= \int_{\tau} \nabla \cdot (\psi \nabla \phi) \differential \tau \\
        &= \int_{\tau} [\psi \nabla^2 \phi + (\nabla \psi) \cdot (\nabla \phi)] \differential \tau
    \end{aligned}
\end{equation*}
\subsection*{Green's Second Theorem}
$\nabla^2$ is a self-adjoint operator in the $L^2$ inner product for functions vanishing on the boundary, equating to zero.
\begin{equation*}
    \begin{aligned}
    \int_{\tau}(\psi \nabla^2 \phi - \phi \nabla^2 &\psi) \differential \tau = \\
    &\int_{S} [\psi(\nabla \phi) - \phi(\nabla \psi)] \cdot \differential \mathrm{S}
    \end{aligned}
\end{equation*}
