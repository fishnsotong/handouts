\section{Vector Calculus Operators}

\subsection*{Gradient}
For a function $\phi(x, y, z)$,
\begin{equation*}
    \nabla \phi = \pdv{\phi}{x} \textbf{\textit{i}} + \pdv{\phi}{y} \textbf{\textit{j}} + \pdv{\phi}{z} \textbf{\textit{k}}
\end{equation*}

\textbf{Directional derivatives.} Where $\textbf{\textit{t}}$ is a \textit{unit vector},
the following equation is the rate of change of $\phi$ with distance in the direction of $\textbf{\textit{t}}$
\begin{equation*}
    \textbf{\textit{t}} \cdot \nabla \phi = \odv{\phi}{s}
\end{equation*}

\textbf{Normal to a surface ($\phi = c$).} The \textit{unit normal vector} to that surface at any point
\begin{equation*}
    \textbf{\textit{n}} = \frac{\nabla \phi}{|\phi|}.
\end{equation*}
Given $\textbf{\textit{n}}$, we can find the tangent plane.
\subsection*{Divergence}
For a vector field $\mathbf{A}$ in $\mathbb{R}^3$
\begin{equation*}
    \nabla \cdot \mathbf{A} = \pdv{A_x}{x} + \smash{\pdv{A_y}{y}}+ \pdv{A_z}{z}
\end{equation*}

Per definition of dot product, for $\mathbb{R}^n$
\begin{equation*}
    \nabla \cdot \mathbf{A} = \sum_{i}^{n} \pdv{A_i}{x_i}
\end{equation*}

$\divergence \mathbf{A}$ is the net flux out per unit volume around a point in a vector field.

\subsection*{Curl}

If $\nabla \times \mathbf{E} = 0$, $\mathbf{E}$ is a conservative vector field. \textit{What does it mean for a field to be conservative?}

\subsection*{Laplacian}
Take note that $\nabla^2 \phi = \divergence \grad \phi$ as we take the inner product of two gradient operators $\nabla \cdot \nabla = \nabla^2$.
\begin{equation*}
    \nabla^2 \phi =  \pdv[order=2]{\phi}{x} + \pdv[order=2]{\phi}{x} + \pdv[order=2]{\phi}{x}
\end{equation*}

$\nabla^2 \phi$ is a useful measure of convexity in higher dimensions.
\vspace{\baselineskip}

For useful identities regarding the above differential operators, see the formula booklet.

\section{Vector Integral Calculus}

\subsection*{Line Integrals}
Line integral over a \textbf{scalar field} is

Line integral over a \textbf{vector field} is

\subsection*{Surface Integrals}
\subsection*{Volume Integrals}
Volume integrals are simpler than line or surface integrals because $\mathrm{d}V$ is a scalar quantity.

\subsection*{Gauss' Theorem (Divergence Theorem)}
\subsection*{Stokes' Theorem}
\subsection*{Green's Theorem}
\subsection*{Green's Second Theorem}