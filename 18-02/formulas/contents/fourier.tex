\columnbreak
\section{Fourier Series}
The trigonometric functions form a \textit{basis} for the space of periodic functions with period $2L$. Almost
any function can be written as the following infinite sum:
\begin{equation*}
    f(x) = \frac{a_0}{2} + \sum_{n=1}^{\infty}\left(a_n \cos\frac{n\pi x}{L} + b_n \sin \frac{n \pi x}{L}\right)
\end{equation*}
The key to finding the Fourier coefficients $a_n$ and $b_n$ is \textit{orthogonality}.

\subsection*{Trigonometric orthogonality relations}
The following results hold over any interval $[a,b]$ of length $2L = b - a \because$ integrands are periodic
with period $2L$.
\begin{equation*}
    \int_{-L}^{L} \cos \frac{m \pi x}{L} \cos \frac{n \pi x}{L} = \begin{cases}
        2L &\textrm{if } m = n = 0 \\
        L  &\textrm{if } m = n \neq 0 \\
        0  &\textrm{if } m \neq n.
    \end{cases}
\end{equation*}

\begin{equation*}
    \int_{-L}^{L} \cos \frac{m \pi x}{L} \sin \frac{n \pi x}{L} = 0
\end{equation*}

\begin{equation*}
    \int_{-L}^{L} \sin \frac{m \pi x}{L} \sin \frac{n \pi x}{L}   = \begin{cases}
        0 \quad \textrm{if } m = n = 0 \\
        L \quad \textrm{if } m = n \neq 0 \\
        0 \quad \textrm{if } m \neq n.
    \end{cases}
\end{equation*}

\subsection*{Finding the Fourier coefficients}
To find $a_0$, integrate f(x) over one period:
\begin{equation*}
    \int_{-L}^{L}f(x) \differential x = \frac{a_0}{2}2L+0+0=a_0L
\end{equation*}
To find $a_m$ for $m \geq 1$, multiply $f(x)$ by $\cos(m \pi x / L)$, integrate over one period:
\begin{equation*}
    \int_{-L}^{L} f(x) \cos \frac{m\pi x}{L} \differential x = 0 + a_m L + 0 = a_m L
\end{equation*}
Only the term $n = m$ contributes to the result. To find $b_m$ for $m \geq 1$, multiply $f(x)$ by $\sin(m \pi x / L)$, integrate over one period:
\begin{equation*}
    \int_{-L}^{L} f(x) \sin \frac{m\pi x}{L} \differential x = 0 + 0 + b_m L = b_m L
\end{equation*}
In summary,
\begin{equation*}
    \begin{aligned}
        a_n &= \frac{1}{L} \int_{-L}^{L}f(x) \cos\frac{n \pi x}{L} \differential x \\
        b_n &= \frac{1}{L} \int_{-L}^{L}f(x) \sin\frac{n \pi x}{L} \differential x
    \end{aligned}
\end{equation*}

\subsection*{Even and odd functions}
Even function: $f(-x) = f(x)$ \\
Odd function: $f(-x) = -f(x)$
\vspace{\baselineskip}

The integral of an even function over a symmetric integral is twice the integral over the positive half of the interval:
\begin{equation*}
    \int_{-L}^{L} f_e(x) \differential x = 2 \int_{0}^{L} f_e(x) \differential x,
\end{equation*}
The integral of an odd function over a symmetric integral is zero:
\begin{equation*}
    \int_{-L}^{L} f_o(x) \differential x = 0.
\end{equation*}

\textbf{When evaluating a Fourier series}, if $f(x)$ is \textit{even}, the Fourier series only contains cosine terms
\begin{equation*}
    f_e(x) = \frac{a_0}{2} + \sum_{n=1}^{\infty} a_n \cos \frac{n \pi x}{L}
\end{equation*}
If f(x) is \textit{odd}, Fourier series only contains sine terms
\begin{equation*}
    f_o(x) =  \sum_{n=1}^{\infty} b_n \sin \frac{n \pi x}{L}
\end{equation*}
Coefficients for the above can be found by integrating over half a period
\begin{equation*}
    \begin{aligned}
        a_n &= \frac{2}{L}\int_{0}^{L} f_e(x) \cos \frac{n \pi x}{L} \differential x \\
        b_n &= \frac{2}{L}\int_{0}^{L} f_o(x) \sin \frac{n \pi x}{L} \differential x
    \end{aligned}
\end{equation*}
Be prepared to integrate by parts, use trigonometric substitutions or symmetry arguments to simplify the integrals.

\subsection*{Parseval's Theorem}
The mean square value of a function $f(x)$ with period $2L$ is:
\begin{equation*}
    \frac{1}{2L}\int_{-L}^{L}[f(x)]^2 \differential x = \frac{a_0^2}{4} = \frac{1}{2}\sum_{n=1}^{\infty}(a_n^2 + b_n^2)
\end{equation*}
Orthogonality relations have eliminated the cross terms of the Fourier series binomial expansion.
\subsection*{Complex Fourier Series}
The Fourier series can be expressed more simply and compactly in the notation of the complex Fourier series.
\begin{equation*}
    f(x) = \sum_{n = -\infty}^{\infty}c_n \exp\left(-\frac{in\pi x}{L}\right)
\end{equation*}
This is useful for the discrete Fourier transform and the fast Fourier transform.

\subsection*{Fourier Transform}
The fourier transform of a signal $f$ is
\begin{equation*}
    F(\omega) = \int_{-\infty}^{\infty} f(t) \exp(-i\omega t) \differential t.
\end{equation*}
$F$ is a function of a real variable $\omega$. 

Generally, the function value of $F(\omega)$ is a complex number.
\begin{equation*}
    \begin{aligned}
        F(\omega) &= A - iB \\
        A &= \int_{-\infty}^{\infty} f(t) \cos(-i\omega t) \differential t \\
        B &= \int_{-\infty}^{\infty} f(t) \sin(-i\omega t) \differential t.
    \end{aligned}
\end{equation*}
