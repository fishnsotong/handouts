\section{Fourier Series}
The trigonometric functions form a \textit{basis} for the space of periodic functions with period $2L$. Almost
any function can be written as the following infinite sum
\begin{equation*}
    f(x) = \frac{a_0}{2} + \sum_{n=1}^{\infty}\left(a_n \cos\frac{n\pi x}{L} + b_n \sin \frac{n \pi x}{L}\right)
\end{equation*}
\subsection*{Parseval's Theorem}
\subsection*{Complex Fourier Series}
The Fourier series can be expressed more simply and compactly in the notation of the complex Fourier series.
\begin{equation*}
    f(x) = \sum_{n = -\infty}^{\infty}c_n \exp\left(-\frac{in\pi x}{L}\right)
\end{equation*}
This is useful for the discrete Fourier transform and the fast Fourier transform.
\subsection*{Fourier Transform}