\section{Partial Derivatives}

For $f(x,y)$ how do we compute $\partial f / \partial x$?

Treat $y$ as constant, and $x$ as variable.

\textbf{Mixed derivatives} can be calculated in any order.
\begin{equation*}
    \pdv{f}{x}{y} = \pdv{f}{y}{x}
\end{equation*}

\subsection*{Second Derivative Test}

A \textit{critical point} of $f(x,y)$ is where \\
$f_x = 0$ \textbf{and} $f_y = 0$.

There are three types of critical point for $f(x,y)$: minima, maxima and \textit{saddle points} - where the graph
of the function lies both above and below the tangent plane at the point.
\vspace{\baselineskip}

1. Find the critical points by solving the simultaneous equations
\begin{equation*}
\begin{cases}
    f_x(x,y) = 0 \\
    f_y(x,y) = 0
\end{cases}
\end{equation*}
2. Calculate the 3 second derivatives at the critical point $(x_0, y_0)$
\begin{equation*}
    A = f_{xx} \quad B = f_{xy} = f_{yx} \quad C = f_{yy}
\end{equation*}

3. \textbf{Second-derivative test.} Three cases.

1. $AC - B^2 > 0$, $A > 0$ or $C > 0 \implies$ min \\
2. $AC - B^2 > 0$, $A < 0$ or $C < 0 \implies$ max \\
3. $AC - B^2 < 0 \implies$ saddle point \\
4. $AC - B^2 = 0 \implies$ fucked

Tangent plane to the graph at the critical point $(x_0, y_0)$ is horizontal.

\subsection*{Polar and Cartesian Coordinates}
Recall that
\vspace{\baselineskip}

$r = \sqrt{x^2 + y^2} \quad \theta = \tan^{-1} (y/x)$ \\
$x = r \cos \theta \quad y = r \sin \theta$ 
\vspace{\baselineskip}

These relations are useful for multiple integrals and vector integrals.