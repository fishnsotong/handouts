\documentclass[a4paper]{tufte-handout}

\title{34. Electronic Spectroscopy \& Photochemistry\thanks{Wayne~W.~Z. Yeo}}

\author[MIT 18.02]{\textnormal{MIT 18.02} Multivariable Calculus\thanks{Course Instructors: Robert Field}}

%\date{28 March 2010} % without \date command, current date is supplied

%\geometry{showframe} % display margins for debugging page layout

\usepackage{graphicx} % allow embedded images
  \setkeys{Gin}{width=\linewidth,totalheight=\textheight,keepaspectratio}
  \graphicspath{{graphics/}} % set of paths to search for images
\usepackage{amsmath,amsthm}  % extended mathematics
\usepackage{physics}
\usepackage[version=4]{mhchem}
\usepackage{booktabs} % book-quality tables
\usepackage{units}    % non-stacked fractions and better unit spacing
\usepackage{multicol} % multiple column layout facilities
\usepackage{lipsum}   % filler text
\usepackage{fancyvrb} % extended verbatim environments
  \fvset{fontsize=\normalsize}% default font size for fancy-verbatim environments
  
  \usepackage{graphicx}
  \usepackage{array,cellspace,multirow}
  \setlength\cellspacetoplimit{5pt}
  \setlength\cellspacebottomlimit{5pt}
  \newcommand\mc[3]{\hfill #1\par          % cell's content at top right
                         (#2)\par          % cell's content on the midle
                           #3\hfill\mbox{}}% cell's content at bottom left

% Standardize command font styles and environments
\newcommand{\doccmd}[1]{\texttt{\textbackslash#1}}% command name -- adds backslash automatically
\newcommand{\docopt}[1]{\ensuremath{\langle}\textrm{\textit{#1}}\ensuremath{\rangle}}% optional command argument
\newcommand{\docarg}[1]{\textrm{\textit{#1}}}% (required) command argument
\newcommand{\docenv}[1]{\textsf{#1}}% environment name
\newcommand{\docpkg}[1]{\texttt{#1}}% package name
\newcommand{\doccls}[1]{\texttt{#1}}% document class name
\newcommand{\docclsopt}[1]{\texttt{#1}}% document class option name

\setlength\fboxsep{7px}

\newenvironment{docspec}{\begin{quote}\noindent}{\end{quote}}% command specification environment

\newtheorem{theorem}{Theorem}
\newtheorem{corollary}{Corollary}
\newenvironment{justification} {\begin{proof}[Justification]} {\end{proof}}

\theoremstyle{definition}
\newtheorem{definition}{Definition}
\newtheorem{example}{Example}


\begin{document}

\maketitle% this prints the handout title, author, and date

\begin{abstract}
\noindent
Vector operations, partial derivatives. Differential operators (div, grad, curl, laplacian). Line integrals, potential fields and Green's Theorem. Surface integrals and Stokes' Theorem. Volume integrals and the Divergence Theorem.

\end{abstract}

\begin{tabular}{cc | *{2}{>{\centering\arraybackslash}S{m{12mm}}|}}
  \multicolumn{2}{c}{}    &   \multicolumn{2}{c}{Player 1}    \\
  \multicolumn{2}{c}{}    &   \multicolumn{1}{c}{A}
                          &   \multicolumn{1}{c}{B}           \\
      \cline{3-4}
  \multirow{4}{*}{\rotatebox[origin = c]{90}{Player 2}}
      &   A   &   \mc{4}{3}{1}    &   \mc{5}{1}{2}            \\
      \cline{3-4}
      &   B   &   \mc{6}{5}{1}    &   \mc{2}{3}{2}            \\
      \cline{3-4}
  \end{tabular}

  \section{Differential Operators (div, grad, curl)}

\subsection{Gradient ($\nabla$)}

$\nabla f$ is called the \textbf{gradient} of $f$.

For a given point, we get a vector\marginnote{$\nabla f$ is a vector-valued function, where $f: \mathbb{R}^n \rightarrow \mathbb{R}$ is a scalar field.}; the
entire operation returns a vector field. Perhaps one of the most important properties of the gradient



\begin{theorem}
  $\nabla f$ is orthogonal to the level surface $w = c$.
\end{theorem}

More stuff nom nom nom.

\begin{equation*}
  \int_{\mathcal{C}}(\nabla \phi) \cdot \mathrm{d} \mathbf{x} = 
\end{equation*}

This result is sometimes called the \textit{gradient theorem}.

\subsection{Divergence ($\nabla \cdot$)}

\subsection{Curl ($\nabla \times$)}

\subsection{Laplacian ($\nabla^2$)}

\section{Stokes' Theorem}

In particular, the line integral of a scalar field is

\begin{theorem}[Stokes' Theorem]

  \begin{equation*}
    \int_{S} (\nabla \times F) \cdot \differential \mathbf{S} = \int_{C} F \cdot \differential \mathbf{x}
  \end{equation*}
  
\end{theorem}

\section{Applications in Electromagnetism}

Vector calculus is an important part 

\begin{theorem}[Gauss' Law]
  \begin{equation*}
    \nabla \cdot \vec{E} = \frac{\rho}{\epsilon_0}
  \end{equation*}
  where $\rho$ is the charge density.
\end{theorem}

By the divergence theorem, we can reformulate the above differential form as the following
\begin{equation*}
  \iint_S \vec{E} \cdot \hat{\mathbf{n}} \differential S = \iiint_D \nabla \cdot \vec{E} \differential V
\end{equation*}
where $Q$ is the total charge inside the closed surface $S$.

This formula tells us how charges influence the electric field.

\bibliography{summary}
\bibliographystyle{plainnat}



\end{document}
