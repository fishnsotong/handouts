
\section{Differentiation}

\subsection*{Preliminaries}

Completing the square, where $ax^2 + bx + c$ can be rewritten as
\begin{equation*}
  a\left(x + \frac{b}{2a}\right)^2 - \left(\frac{b}{2a}\right)^2 + c
\end{equation*}


For a linear function $Y = mX + C$, an equation can be formed with $y - y_1 = m(x - x_1)$.

\subsection*{Definition and Continuity}
\begin{equation*}
  f'(x) = \lim_{\Delta x \rightarrow 0} \frac{f(x + \Delta x) - f(x)}{\Delta x}
\end{equation*}

\textbf{diff $\Rightarrow$ cont. }If $f$ is differentiable at $x_0$, then $f$ is continuous at $x_0$. 

Contrapositively, if $f$ is discontinuous at $x_0$, then $f$ is not differentiable at $x_0$

\subsection*{General and Specific Formulae}

$(u + v)' = u' + v'$

$(cu)' = cu'$

$(uv)' = u'v + v'u$ (product rule)

$(u/v)' = (u'v - v'u) / v^2$ (quotient rule)

$[f(u(x))]' = f'(u(x)) \cdot u'(x)$ (chain rule)

which can be written altneratively below,

$$\frac{\mathrm{d}f}{\mathrm{d}x} = \frac{\mathrm{d}f}{\mathrm{d}u} \frac{\mathrm{d}u}{\mathrm{d}x}$$

For specific formulae, refer to tables.

\subsection*{Implicit Differentiation}
Instead of solving for $y$ and then taking its derivative, you can take the derivative of both sides of the equation.

\subsection*{Important Trigonometric Limits}
These allow us to find the derivatives of trig functions (l'Hopital's rule).

$\lim_{\theta \rightarrow 0} \sin \theta / \theta \rightarrow 0$ 

$\lim_{\theta \rightarrow 0} (1 - \cos \theta) / \theta \rightarrow 1$ \\

Evaluating the value of $e$:
\begin{equation*}
  \lim_{k \rightarrow \infty} \left(1 + \frac{1}{k}\right)^k = e
\end{equation*}

\subsection*{Hyperbolic Functions}
Regular trig functions are `circular' functions. If $u = \cos(x)$ and $v = \sin(x)$, then $u^2 + v^2 = 1$, which is the equation of a unit circle.

\section{Applications of Differentiation}