
\section{Differentiation}

\subsection*{Preliminaries}

Completing the square

For a linear function $Y = mX + C$, an equation can be formed with $y - y_1 = m(x - x_1)$.

\textbf{Photon Energy ($E$)}. $E = hf$ where
$h$ is Planck's constant and $f$ is the frequency of the electromagnetic
radiation.
\vspace{\baselineskip}

\textbf{de Broglie Wavelength.} The wavelength of a particle is given by
\begin{equation*} \lambda = \frac{h}{p} = \frac{h}{mv} \end{equation*} where $p$
is the momentum of the object.
\vspace{\baselineskip}

\textbf{Heisenberg's Uncertainty Principle.} \begin{equation*} \Delta x \Delta p \geq \frac{\hslash}{2} =
  \frac{h}{4\pi} \end{equation*} where $\hslash$ is the reduced Planck's constant.
\vspace{\baselineskip}

\textbf{Electronvolt} (eV). Useful unit of energy where $1 \textrm{ eV} = 1.60 \times 10^{-19} \textrm{ J}$.
\vspace{\baselineskip}

\textbf{Wavenumbers}. $\nu(\mathrm{cm}^{-1}) = 10^7 / \lambda (\mathrm{nm})$
\vspace{\baselineskip}

\subsection*{Schrödinger Equation}
Time Independent Schrödinger Equation
\begin{equation*}
    \hat{H}\psi = E\psi
\end{equation*}

\subsection*{Operators}
Position operator: $\hat{x} = x$ \\
Momentum operator: $\hat{p}_x = \hbar/i \cdot (\mathrm{d}/\mathrm{d}x)$ \\
Hamiltonian operator: $\hat{H} = \hat{E}_k + \hat{V}$
\vspace{\baselineskip}

\subsection*{Wavefunctions}
$\psi$ must satisfy the following conditions.
\begin{itemize}
    \item Not infinite over a finite range
    \item Single-valued
    \item Continuous everywhere
    \item Continuous first derivative
\end{itemize}

\subsection*{Free Translation in 1D}
Schrödinger Equation:
$$\frac{-\hbar^2}{2m} \frac{\mathrm{d}^2\psi(x)}{\mathrm{d}x^2} =  E \psi(x)$$

Wavefunctions and energies:
$$\psi(x) = A {\rm e}^{ikx} + B {\rm e}^{-ikx} \quad E_k = \frac{k^2\hbar^2}{2m}$$


\subsection*{Particle in a Box}
Hamiltonian:
$$\hat{H} = \frac{-\hbar^2}{2m} \frac{\mathrm{d}^2}{\mathrm{d}x^2} + V(x)$$

\subsection*{Particle in a Ring}
Hamiltonian:
$$\hat{H} = \frac{-\hbar^2}{2mr^2} \frac{\mathrm{d}^2}{\mathrm{d}\phi^2}$$

\subsection*{Harmonic Oscillator}
Hamiltonian:
$$\hat{H} = \frac{-\hbar^2}{2m} \frac{\mathrm{d}^2}{\mathrm{d}x^2} + \frac{1}{2}kx^2$$

\subsection*{Hydrogen Atom}
Electronic Hamiltonian of \ce{H}:
$$\hat{H}_e = \frac{-\hbar^2}{2m_e}\nabla_e^2-\frac{e^2}{4\pi\epsilon_0} \left( \frac{1}{r_A} \right)$$

\section{Many-Electron Systems}
\subsection*{Born-Oppenhimer Approximation}

Hamiltonian for a many-electron system:
$$\hat{H} = \hat{T}_{n} + \hat{T}_{e} + \hat{V}_{ee} + \hat{V}_{en} + \hat{V}_{nn}$$

Born-Oppenheimer Hamiltonian: $\hat{T} = 0$,
$$\hat{H}_{\mathrm{BO}} = \left( \hat{T}_{e} + \hat{V}_{ee} + \hat{V}_{en} \right) + \hat{V}_{nn}$$

\subsection*{Hydrogen Ion \ce{H2+}}

Electronic Hamiltonian of \ce{H2+}
\begin{equation*}
    \hat{H}_e = \frac{-\hbar^2}{2m_e}\nabla^2_e + \frac{e^2}{4\pi\epsilon_0} \left( \frac{1}{R_{AB}} - \frac{1}{r_{A}} - \frac{1}{r_{B}} \right)
\end{equation*}

A \textbf{basis set} is the set of atomic orbitals from which molecular orbitals are constructed. A
\textbf{minimal basis set} is the \textit{smallest number} of AOs needed to create MOs that can
accommodate all electrons in a molecule.
\vspace{\baselineskip}

Molecular Orbitals of \ce{H2+}
\begin{equation*}
    \psi(\ce{H2+}) = c_A[1\ce{s_}_A (\mathbf{r}_A) \pm 1\ce{s_}_B (\mathbf{r}_B)]
  \end{equation*}

$1\ce{s_}_A (\mathbf{r}_A) + 1\ce{s_}_B (\mathbf{r}_B) \rightarrow$ constructive \\
$1\ce{s_}_A (\mathbf{r}_A) - 1\ce{s_}_B (\mathbf{r}_B) \rightarrow$ destructive 

$$c_A = \frac{1}{\sqrt{2 (1 \pm S_{AB})}}$$

$c_A$ is determined by normalization

Substituting in the values of $c_A$,

$$\psi(\ce{H2+}) = \frac{[1\ce{s_}_A (\mathbf{r}_A) \pm 1\ce{s_}_B (\mathbf{r}_B)]}{\sqrt{2 (1 \pm S_{AB})}}$$

Energies of \ce{H2+} Molecular Orbitals

$$E = \frac{(H_{AA} \pm H_{AB})}{1 \pm S_{AB}}$$

can approximate this as $E_{\pm} = E_{1s} \pm H_{AB}$.

\textbf{Overlap integral} $S_{AB}$ -- extent of orbital overlap, typically $S_{AB} \ll 1$.

\textbf{Coulomb integral} $H_{AA}$ -- energy of an electron in \ce{1s} orbital of $A$, with the added electrostatic potential of proton $B$.

\textbf{Exchange integral} $H_{AB}$ -- electron is not restricted spatially to a \ce{1s} orbital on $A$ or $B$,
but can exchange between both atomic orbitals.
\vspace{\baselineskip}

\subsection*{Variational Theorem}
The \textbf{variational principle} provides a criterion for optimizing a trial wavefunction.

Improve the quality of minimal basis LCAO approximation by using a scaling parameter $\psi_{1s} = \exp (-Zr_1/a_0)$

Can add more basis functions to describe where the electron should be.

\subsection*{Indistinguishable Electrons}
Equivalent configurations of many indistinguishable particles occur with equal probability.

If we exchange two indistinguishable particles, then at most the
wave function can change sign. 

\subsection*{Space and Spin}
Electrons always have \textit{antisymmetric} wavefunctions -- they are fermions.
$$\psi = \psi_{space} \times \psi_{spin}$$

Therefore, only two such configurations are possible for \textbf{spin wavefunctions}.

$$ \psi^{\mathrm{AS}} = \psi_{space}^{\mathrm{S}} \times \psi_{spin}^{\mathrm{AS}} \quad \psi^{\mathrm{AS}} = \psi_{space}^{\mathrm{AS}} \times \psi_{spin}^{\mathrm{S}} $$

For the \ce{H2} molecule:

$\alpha(1)\alpha(2)$ -- S, triplet

$\beta(1)\beta(2)$ -- S, triplet

$(1/\sqrt{2})[\alpha(1)\beta(2)+\beta(1)\alpha(2)]$ -- S, triplet

$(1/\sqrt{2})[\alpha(1)\beta(2)-\beta(1)\alpha(2)]$ -- A, singlet

Lecture 6: evaluate the exchange energy $\therefore$ triplet states have lower energies than the corresponding singlet states