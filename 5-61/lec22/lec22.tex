\documentclass[a4paper]{tufte-handout}

\title{22. The Helium Atom\thanks{Wayne~W.~Z. Yeo}}

\author[MIT 5.61]{\textnormal{MIT 5.61} Physical Chemistry\thanks{Course Instructors: Robert Field, with lecture notes from Troy van Voorhis}}

%\date{28 March 2010} % without \date command, current date is supplied

%\geometry{showframe} % display margins for debugging page layout

\usepackage{graphicx} % allow embedded images
  \setkeys{Gin}{width=\linewidth,totalheight=\textheight,keepaspectratio}
  \graphicspath{{graphics/}} % set of paths to search for images
\usepackage{amsmath,amsthm}  % extended mathematics
\usepackage{physics}
\usepackage[version=4]{mhchem}
\usepackage{booktabs} % book-quality tables
\usepackage{units}    % non-stacked fractions and better unit spacing
\usepackage{multicol} % multiple column layout facilities
\usepackage{lipsum}   % filler text
\usepackage{fancyvrb} % extended verbatim environments
  \fvset{fontsize=\normalsize}% default font size for fancy-verbatim environments

% Standardize command font styles and environments
\newcommand{\doccmd}[1]{\texttt{\textbackslash#1}}% command name -- adds backslash automatically
\newcommand{\docopt}[1]{\ensuremath{\langle}\textrm{\textit{#1}}\ensuremath{\rangle}}% optional command argument
\newcommand{\docarg}[1]{\textrm{\textit{#1}}}% (required) command argument
\newcommand{\docenv}[1]{\textsf{#1}}% environment name
\newcommand{\docpkg}[1]{\texttt{#1}}% package name
\newcommand{\doccls}[1]{\texttt{#1}}% document class name
\newcommand{\docclsopt}[1]{\texttt{#1}}% document class option name

\setlength\fboxsep{7px}

\newenvironment{docspec}{\begin{quote}\noindent}{\end{quote}}% command specification environment

\newtheorem{theorem}{Theorem}
\newtheorem{corollary}{Corollary}
\newenvironment{justification} {\begin{proof}[Justification]} {\end{proof}}

\theoremstyle{definition}
\newtheorem{definition}{Definition}
\newtheorem{example}{Example}


\begin{document}

\maketitle% this prints the handout title, author, and date

\begin{abstract}
\noindent
We now move toward a discussion of many-electron atoms by setting up an approximate description of the
simplest example: the helium atom, (He $Z = 2$). \textbf{Finding an exact solution is impossible.} We then
consider excited states and electron pairing.

\end{abstract}

Now that we have treated hydrogen-like atoms in some detail,

It turns out to be fairly difficult to transform to the centre of mass when dealing with three particles.

\section{Quantum Mechanics of Many Particles}

We will treat the quantum mechanics of multiple particles in much the same way as we described multiple dimensions. \marginnote{Why can we do that?}We
will invent position operators $\hat{\mathbf{r}}_1, \hat{\mathbf{r}}_2, \hat{\mathbf{r}}_3, \dots$ and associated momentum operators $\hat{\mathbf{p}}_1, \hat{\mathbf{p}}_2, \hat{\mathbf{p}}_3, \dots$
The operators for a given particle ($i$) will be assumed to commute with all operators associated with any other particle ($j$):

\begin{equation*}
  [\hat{\mathbf{r}}_1, \hat{\mathbf{p}}_2] = [\hat{\mathbf{p}}_2, \hat{\mathbf{r}}_3] = [\hat{\mathbf{r}}_2, \hat{\mathbf{r}}_3] = [\hat{\mathbf{p}}_1, \hat{\mathbf{p}}_3] = \dots = 0.
\end{equation*}

Meanwhile, operators belonging to the same particle will obey the normal commutation relations.\marginnote{Reminiscent of matrix multiplication, thanks Heisenberg!} Position and momentum along a given axis
do not commute:

\begin{equation*}
  [\hat{x}_1, \hat{\mathbf{p}}_{x1}] = i\hbar \quad [\hat{y}_2, \hat{\mathbf{p}}_{y2}] = i \hbar \quad [\hat{y}_2, \hat{\mathbf{p}}_{y2}] = i \hbar
\end{equation*}\marginnote{As you can already see, one of the biggest challenges to treating multiple electrons is the explosion in the
number of variables required!}In terms of the operators described above, we can quickly write the Hamiltonian for the helium atom.

\begin{equation}
  \hat{H} = \frac{\hat{\mathbf{p}}_1^2}{2m_e} + \frac{\hat{\mathbf{p}}_2^2}{2m_e} - \frac{Ze^2}{4\pi\epsilon_0 \hat{\mathbf{r}}_1} - \frac{Ze^2}{4\pi\epsilon_0 \hat{\mathbf{r}}_2} + \frac{e^2}{4\pi\epsilon_0 | \hat{\mathbf{r}}_1 - \hat{\mathbf{r}}_2|}
\end{equation} 

This Hamiltonian looks very intimidating, mainly because of all the constants ($e, m_e, e_0$, etc.) that appead in the equation. It is therefore
much simpler to work everything out in atomic units.

\section{Non-Interacting Electron Approximation}

For He, the first thing we notice is that the Hamiltonian becomes seperable if we neglect the
electron-electron repulsion term:

\section{Independent Electron Approximation}

\begin{theorem}[Spin-Statistics Theorem] Integer-spin particles are bosons, while half-integer-spin particles are fermions.
  \marginnote{The fields of integral spins commute (and therefore must be quantised as bosons) while the fields of
  half-integral spins anticommute (and therefore must be quantised as fermions).} 
  \begin{enumerate}
    \item Any wavefunction that describes multiple identical fermions must be antisymmetric upon exchange of the electrons.
    \item Any wavefunction that describes multiple identical bosons must be symmetric upon exchange of the particles
  \end{enumerate}
   
\end{theorem}

\section{Energies of Singlet and Triplet States}

Thus, in terms of $J$ and $K$ the energies of the singlet and triplet states become:
\begin{equation*}
  E_{S/T} = E_{1\ce{s}} + E_{2\ce{s}} + J_{1\ce{s}2\ce{s}} \pm  K_{1\ce{s}2\ce{s}}
\end{equation*}

In other chemistry courses you may hev been taught that there are only two spin states for an electron pair: $\uparrow \downarrow$ and $\uparrow \uparrow$. The
former represented the singlet (S) state and the latter the triplet (T) state.

Instead, they look like $\uparrow \downarrow \pm \downarrow \uparrow$, with $\uparrow \downarrow - \downarrow \uparrow$ being the singlet and
$\uparrow \downarrow + \downarrow \uparrow$ being part of the triplet.

\subsection*{What determines electron pairing?}

In the more precise picture we have derived here, \textbf{the \textit{spin} part of the wavefunction determines \marginnote{Electron spins are not labeled as they are indistinguishable.}
if electrons are paired or unpaired}. An electron pair has the characteristic spin part $\alpha \beta - \beta \alpha$; paired 
electrons form a singlet. Spin parts that look like $\alpha \alpha$, $\beta \beta$ or $\alpha \beta + \beta \alpha$ are unpaired triplet configurations.

As we have seen above, \marginnote{In some situations, this is called the pairing energy.} \textbf{pairing two electrons raises the energy through the exchange integral}. The
counterintuitive thing that we have to re-learn is that $\alpha \beta + \beta \alpha$ does not describe an electron pair.

Given this picture, we note that the three triplet states would then correspond to three possible $z$-projections of spin. The three triplet states should then have
\begin{equation*}
  M_s = -1, 0, +1.
\end{equation*}

This gives us some qualitative picture of what the $\alpha \beta + \beta \alpha$ state means and why it corresponds to unpaired electrons. In the
$\alpha \beta + \beta \alpha$ state, the spins are oriented parallel to each other, but they are both oriented \textit{perpendicular} to the $z$-axis.

\bibliography{lec34}
\bibliographystyle{plainnat}

\end{document}
