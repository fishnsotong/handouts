\documentclass[a4paper]{tufte-handout}

\title{3. Photochemistry\thanks{Wayne~W.~Z. Yeo}}

\author[ESB]{\textnormal{CHEM 50003} Electronic States and Bonding\thanks{Course Instructors: Robert Field}}

%\date{28 March 2010} % without \date command, current date is supplied

%\geometry{showframe} % display margins for debugging page layout

\usepackage{graphicx} % allow embedded images
  \setkeys{Gin}{width=\linewidth,totalheight=\textheight,keepaspectratio}
  \graphicspath{{graphics/}} % set of paths to search for images
\usepackage{amsmath,amsthm}  % extended mathematics
\usepackage{physics}
\usepackage[version=4]{mhchem}
\usepackage{booktabs} % book-quality tables
\usepackage{units}    % non-stacked fractions and better unit spacing
\usepackage{multicol} % multiple column layout facilities
\usepackage{lipsum}   % filler text
\usepackage{fancyvrb} % extended verbatim environments
  \fvset{fontsize=\normalsize}% default font size for fancy-verbatim environments

% Standardize command font styles and environments
\newcommand{\doccmd}[1]{\texttt{\textbackslash#1}}% command name -- adds backslash automatically
\newcommand{\docopt}[1]{\ensuremath{\langle}\textrm{\textit{#1}}\ensuremath{\rangle}}% optional command argument
\newcommand{\docarg}[1]{\textrm{\textit{#1}}}% (required) command argument
\newcommand{\docenv}[1]{\textsf{#1}}% environment name
\newcommand{\docpkg}[1]{\texttt{#1}}% package name
\newcommand{\doccls}[1]{\texttt{#1}}% document class name
\newcommand{\docclsopt}[1]{\texttt{#1}}% document class option name

\setlength\fboxsep{7px}

\newenvironment{docspec}{\begin{quote}\noindent}{\end{quote}}% command specification environment

\newtheorem{theorem}{Theorem}
\newtheorem{corollary}{Corollary}
\newenvironment{justification} {\begin{proof}[Justification]} {\end{proof}}

\theoremstyle{definition}
\newtheorem{definition}{Definition}
\newtheorem{example}{Example}


\begin{document}

\maketitle% this prints the handout title, author, and date

\begin{abstract}
\noindent
Dealing with multi-electron systems require more approximations, including the \textbf{Born-Oppenheimer approximation}.

\end{abstract}

\begin{enumerate}
  \item Allowed electronic transitions must involve a change symmetry.
  \item The transition dipole moment is a one-electron operator.
  \item Allowed electronic transitions preserve spin.
  \item Electronic transitions are always accompanied by vibrational transitions. [Franck-Condon Principle]
\end{enumerate}

%\printclassoptions

In general, we see that \textbf{allowed electronic transitions must involve a change in symmetry}. \cite{atkins2014atkins}

At this point, we have nearly completed our introduction to quantum mechanics 
and we can investigate the electronic structure of molecules.

The Hamiltonian of \ce{H2+} is expressed as follows:

\begin{equation}
  \hat{H} = -\frac{1}{2}\nabla^2_r - \frac{\nabla^2_A}{2M_A}
\end{equation}

\section{The Frank-Condon Principle}

\begin{theorem}[The Variational Theorem] $E_{\mathrm{avg}} > E_0$ for any function $\psi$. \marginnote{This
  makes physical sense, because energy of the ground state is, by definition, the lowest possible energy.}

  \begin{proof}
    Expand $\psi$ as a linear combination of unknown eigenstates $\phi_n$ of the Hamiltonian:
    \begin{equation*}
      \psi = \sum_n a_n \phi_n
    \end{equation*}
  \end{proof}
  
\end{theorem}

The variational method does two things for us. First, it gives us a way to compare two 
different wavefunctions and to show which one is closer to the wavefunction of the ground state. 

\section{Bimolecular Photochemistry}

\begin{equation}
  \frac{\phi_{F}}{\phi_{F}'} = \frac{\tau_0}{\tau} = \frac{k_0 + k_Q[Q]}{k_0}
\end{equation}

\section{Fluoresence Resonance Energy Transfer}

\bibliography{photochem}
\bibliographystyle{plainnat}



\end{document}
