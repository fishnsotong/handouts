\documentclass[a4paper]{tufte-handout}

\title{2. Qualitative Molecular Orbital Theory \thanks{Wayne~W.~Z. Yeo}}

\author[ESB]{Electronic States and Bonding\thanks{Course Instructors: Michael Bearpark, Joāo Pedro Malhaldo, Andreas Kafizas}}

%\date{28 March 2010} % without \date command, current date is supplied

%\geometry{showframe} % display margins for debugging page layout

\usepackage{graphicx} % allow embedded images
  \setkeys{Gin}{width=\linewidth,totalheight=\textheight,keepaspectratio}
  \graphicspath{{graphics/}} % set of paths to search for images
\usepackage{amsmath,amsthm,mathrsfs}  % extended mathematics
\usepackage[version=4]{mhchem}        % extended chemistry
\usepackage{xcolor}   % colors (like highlighting)
\usepackage{booktabs} % book-quality tables
\usepackage{units}    % non-stacked fractions and better unit spacing
\usepackage{multicol} % multiple column layout facilities
\usepackage{lipsum}   % filler text
\usepackage{fancyvrb} % extended verbatim environments
  \fvset{fontsize=\normalsize}% default font size for fancy-verbatim environments

% Standardize command font styles and environments
\newcommand{\doccmd}[1]{\texttt{\textbackslash#1}}% command name -- adds backslash automatically
\newcommand{\docopt}[1]{\ensuremath{\langle}\textrm{\textit{#1}}\ensuremath{\rangle}}% optional command argument
\newcommand{\docarg}[1]{\textrm{\textit{#1}}}% (required) command argument
\newcommand{\docenv}[1]{\textsf{#1}}% environment name
\newcommand{\docpkg}[1]{\texttt{#1}}% package name
\newcommand{\doccls}[1]{\texttt{#1}}% document class name
\newcommand{\docclsopt}[1]{\texttt{#1}}% document class option name

\setlength\fboxsep{7px}

\newenvironment{docspec}{\begin{quote}\noindent}{\end{quote}}% command specification environment

\newtheorem{theorem}{Theorem}
\newtheorem{corollary}{Corollary}
\newenvironment{justification} {\begin{proof}[Justification]} {\end{proof}}

\theoremstyle{definition}
\newtheorem{definition}{Definition}
\newtheorem{example}{Example}

\begin{document}

\maketitle% this prints the handout title, author, and date

\begin{abstract}
\noindent
Symmetry elements and operations. QMOT. SALC.
\end{abstract}

%\printclassoptions

%\newthought{Even the simplest molecule}, \ce{H2+} consists of three particles, and its

\section{Symmetry Elements}

\subsection{Reading Character Tables}
Each point group has its own \textbf{character table}.

\begin{example}[Finding irreducible representations of the central atom]
  Here, we find the symmetry labels for the \ce{p} and \ce{d} orbitals of \ce{Pt},
  the central atom in \ce{[PtCl4]^{2-}}.

  \begin{enumerate}
    \item First, you have to know the geometry of your compound. For example, \ce{[PtCl4]^{2-}} is square planar.
    \item The next step is to determine the point group of the compound from this geometry by identifying the symmetry elements that it posesses. \textit{Use a point group flow chart}. The point group of the \ce{[PtCl4]^{2-}} complex is $D_{4\mathrm{h}}$.
    \item Now, consult the character table of the point group we determined in the previous step.
    The character table of the $D_{4\mathrm{h}}$ point group is shown here:
  \end{enumerate}

  $$\begin{array}{c|cccccccccc|cc} \hline
  D_\mathrm{4h} & E & 2C_4 & C_2 & 2C_2' & 2C_2'' & i & 2S_4 & \sigma_\mathrm{h} & 2\sigma_\mathrm{v} & 2\sigma_\mathrm{d} & & \\ \hline
  \mathrm{A_{1g}} & 1 & 1 & 1 & 1 & 1 & 1 & 1 & 1 & 1 & 1 & & x^2+y^2,z^2 \\
  \mathrm{A_{2g}} & 1 & 1 & 1 & -1 & -1 & 1 & 1 & 1 & -1 & -1 & R_z & \\
  \mathrm{B_{1g}} & 1 & -1 & 1 & 1 & -1 & 1 & -1 & 1 & 1 & -1 & & x^2-y^2 \\
  
  \mathrm{B_{2g}} & 1 & -1 & 1 & -1 & 1 & 1 & -1 & 1 & -1 & 1 & & xy \\
  \mathrm{E_g} & 2 & 0 & -2 & 0 & 0 & 2 & 0 & -2 & 0 & 0 & (R_x,R_y) & (xz,yz) \\
  \mathrm{A_{1u}} & 1 & 1 & 1 & 1 & 1 & -1 & -1 & -1 & -1 & -1 & & \\
  \mathrm{A_{2u}} & 1 & 1 & 1 & -1 & -1 & -1 & -1 & -1 & 1 & 1 & z & \\
  \mathrm{B_{1u}} & 1 & -1 & 1 & 1 & -1 & -1 & 1 & -1 & -1 & 1 & & \\
  \mathrm{B_{2u}} & 1 & -1 & 1 & -1 & 1 & -1 & 1 & -1 & 1 & -1 & & \\
  \mathrm{E_u} & 2 & 0 & -2 & 0 & 0 & -2 & 0 & 2 & 0 & 0 & (x,y) & \\ \hline
  \end{array}$$

  Since the three p-orbitals point directly along the $x$, $y$, and $z$ axes, they show the same transformational properties as the basis vectors,
  and consequently get the same symmetry labels, i.e.

  \begin{align*}
  \left.\begin{array}{c}
  \mathrm{p}_x \\ \mathrm{p}_y \end{array}\right\} &\rightarrow \mathrm{E_u} \\
  \mathrm{p}_z &\rightarrow \mathrm{A_{2u}} \ .
  \end{align*}
  
  The five d-orbitals show the same transformational properties as their quadratic function counterparts and thus they get the symmetry labels, i.e.
  
  \begin{align*}
  \mathrm{d}_{xy} &\rightarrow \mathrm{B_{2g}} \\
  \left.\begin{array}{c}
  \mathrm{d}_{xz} \\ \mathrm{d}_{yz} \end{array}\right\} &\rightarrow \mathrm{E_g} \\
  \mathrm{d}_{z^2} &\rightarrow \mathrm{A_{1g}} \\
  \mathrm{d}_{x^2 - y^2} &\rightarrow \mathrm{B_{1g}} \ .
  \end{align*}
  
\end{example}

\subsection{Reading Mulliken Labels}

\begin{tabular}{ l l l }
  \hline
    \ce{A} & &singly degenerate, symmetric about $C_n$ axis ($+1$ in tables)  \\

    \ce{B} & &singly degenerate, antisymmetric about $C_n$ axis ($-1$ in tables)  \\

    \ce{E} & &doubly generate  \\

    \ce{T} & &triply generate  \\
    subscript 1 & e.g. \ce{A1} &symmetric about $C_2$ axis $\perp$ to $C_n$ axis or $\sigma_v$ if no $C_2$ present \\
    subscript 2 & e.g. \ce{A2} &antisymmetric about $C_2$ axis $\perp$ to $C_n$ axis or $\sigma_v$ if no $C_2$ present \\
    g & e.g. \ce{E_{2}}$_{\mathrm{g}}$ &\textit{gerade}, symmetric to inversion $i$ \\
    u & e.g. \ce{B_{2}}$_{\mathrm{u}}$ &\textit{ungerade}, antisymmetric to inversion $i$ \\
    superscript ' & e.g. \ce{A}' &symmetric to $\sigma_h$ \\
    superscript '' & e.g. \ce{E}'' &antisymmetric to $\sigma_h$ \\
  \hline
  \end{tabular}

\section{Symmetry Adapted Linear Combinations}

\subsection{Reducible Representations}
When we construct \textbf{symmetry adapted linear combinations} to build an MO diagram, we choose the $z$
axis as the \textit{principal axis}
\sidenote{Almost all character tables follow the conventions stated in Mulliken's paper "Report on Notation for the Spectra of Polyatomic Molecules" \textit{The Journal of Chemical Physics} 23, \textbf{1997} (1955).}
 (the axis of highest symmetry). 

\section{Appendix A: Symmetry Transformations}

\begin{definition}[Symmetry Transformation]
  A symmetry transformation is a change in our point of view that does not change the results of possible experiments.
  If an observer $O$ sees a system in a state represented by a ray $\mathscr{R}$ or $\mathscr{R}_{1}$ or $\mathscr{R}_{2} \ldots,$ then an equivalent observer $O^{\prime}$ who looks at the same system will observe it in a different state, represented by a ray $\mathscr{R}^{\prime}$ or $\mathscr{R}_{1}^{\prime}$ or $\mathscr{R}_{2}^{\prime} \ldots,$ respectively, but the two observers must find the same probabilities
$$
P\left(\mathscr{R} \rightarrow \mathscr{R}_{n}\right)=P\left(\mathscr{R}^{\prime} \rightarrow \mathscr{H}_{n}^{\prime}\right)
$$
\end{definition}




\bibliography{quantum}
\bibliographystyle{plainnat}



\end{document}
