\section{Photophysics}
\textbf{Franck-Condon Principle}. $\because$ nuclei are much more massive than electrons,
an electronic transition (electron transfer, absorption) takes place much faster than the nuclei can respond.

\subsection*{Fermi's Golden Rule}
$$k(\mathrm{s}^{-1}) = \frac{2\pi}{h}\ce{FC} \times \mathrm{V}^2$$

$\mathrm{V}$ is the perturbation matrix element (electric dipole), and $\ce{FC}$ is the Franck-Condon Factor $|S(f,i)|^2$.
\vspace{\baselineskip}

\textbf{Frank-Condon Factor}: The square of the overlap integral between the \textit{vibrational} wavefunctions.
$$|S(f,i)|^2 = \left( \int^{\infty}_{-\infty}\phi_i\phi_f \mathrm{d}x \right)^2$$

The FC factor will be largest when there is \textit{good overlap} between two vibrational wavefunctions. Principal maxima of
the wavefunctions should be aligned. $0 \rightarrow 0$ transitions are strongest when PE curves are similar.
\vspace{\baselineskip}

\section{Light Absorption}
\textbf{Absorbance}: $$ A = -\log \frac{I}{I_0}$$ \\
$I$ is the intensity of transmitted light, and $I_0$ is the intensity of incident light

\vspace{\baselineskip}

\textbf{Beer Lambert Law}: $A = \epsilon cl$
\vspace{\baselineskip}

\textbf{Oscillator Strength} ($f$) is the empirical `probability' of absorption / emission of photon in electronic transitions.
\begin{equation*}
  f = 4 \times 10^{-9} \int_{\nu_1}^{\nu_2}\epsilon \mathrm{d}\nu \approx 4 \times 10^{-9} \epsilon_{\mathrm{max}} \Delta \nu_{1/2}
\end{equation*}

$f = 1$ is the `fully allowed' transition.

$\Delta \nu_{1/2}$ is the full-width at half maximum of the peak, converted to wavenumbers.
\vspace{\baselineskip}

\textbf{Transition Dipole}
\begin{equation*}
  \mathbf{\mu}_{fi} = -e \int_{-\infty}^{\infty}\psi^{*}_{f} x \psi_{i} \mathrm{d}x
\end{equation*}

Transition dipole is related to oscillator strength: $|\mu_{fi}|^2 \propto f$.
\vspace{\baselineskip}

\textbf{Selection Rules} 1. spin 2. symmetry \\
$\because$ conservation of angular momentum, $\Delta S =0 \therefore \mathrm{S_0} \not\rightarrow \mathrm{T_1} \wedge \mathrm{S_0} \rightarrow \mathrm{S_1}$

For centrosymmetric molecules, only $u \rightarrow g$ and  $g \rightarrow u$ transitions are allowed.
\vspace{\baselineskip}
\section{Light Emission}

Absorption:\\ $\ce{S} + h\nu_i \rightarrow \ce{S}^{*}\quad \nu_{\mathrm{abs}} = I_{\mathrm{abs}}$\\
Fluorescence:\\ $\ce{S}^{*} \rightarrow \ce{S} + h\nu_f \quad \nu_{\mathrm{F}} = k_{\mathrm{F}} [\ce{S}^{*}]$ \\
Internal conversion:\\ $\ce{S}^{*} \rightarrow \ce{S} \quad \nu_{\mathrm{IC}} = k_{\mathrm{IC}}[\ce{S}^{*}]$ \\
Intersystem crossing:\\ $\ce{S}^{*} \rightarrow \ce{T} \quad \nu_{\mathrm{ISC}} = k_{\mathrm{ISC}}[\ce{S}^{*}]$ \\

\subsection*{Fluorescence}
Stokes Shift:
1. fast vibrational relaxation 2. solvent reorganization
\vspace{\baselineskip}

Strickler-Berg Relationship: can relate fluoresence rate $k_R$ with empirical absorption strength (area under curve)
$$ k_R (\mathrm{s}^{-1})= 3 \epsilon_{\mathrm{max}} \Delta \nu_{1/2}$$

Differential Rate Law:
$$\frac{\mathrm{d}[\ce{S}_1]}{\mathrm{d}t} = -(k_R +k_{IC} +k_{ISC}) [\ce{S}_1]$$

Integrated Rate Law:
$$[\ce{S}_1]_t = [\ce{S}_1]_0 \exp[-(k_R + k_{IC} + k_{ISC})t]$$

Fluorescence Quantum Yield:
$$\mathbf{\Phi}_F = \frac{k_R}{k_R + k_{IC} + k_{ISC}} = \frac{k_R}{k_0} = \frac{\tau_0}{\tau_R}$$

\subsection*{Phosphorescence}

Intersystem crossing (ISC) is possible due to \textbf{spin-orbit coupling}: the coupling of \ce{e^-} spin
with orbital angular momentum.

ISC is encouraged by heavy atoms and large molecules with densely populated vibrational states.
\vspace{\baselineskip}

Triplet Formation Quantum Yield:
$$\mathbf{\Phi}_T = \frac{k_{ISC}}{k_0}$$

\section{Unimolecular Photochemistry}

\textbf{Rate equation} \\
Unimolecular: $\ce{S^*} \rightarrow \ce{P} \quad \nu_{\mathrm{uni}} = k_{\mathrm{uni}}[\ce{S^*}]$
\vspace{\baselineskip}

\textbf{Examples} \\
Ionisation: $\ce{A^*} \rightarrow \ce{A+} + \ce{e-}$ \\
Unimolecular dissociation: $\ce{A^{*}} \rightarrow \ce{B} + \ce{C}$ \\
Isomerisation: $\ce{A^*} \rightarrow \ce{A}'$ \\

$\phi_F$ and $\tau_0$ are quenched: photochemistry is an additional first-order process.
$$\tau_0' = \frac{1}{k_0 + k_{PC}} \qquad \phi_f' = \frac{k_R}{k_0 + k_{PC}}$$

Photochemical Quantum Yield:
$$\phi_{PC} = \frac{k_{PC}}{k_0 + k_{PC}} = 1 - \frac{\phi_F'}{\phi_F}$$

Photochemical Rate Constant:
$$k_{PC} = k_R \left( \frac{1}{\phi_F'} - \frac{1}{\phi_F}\right) = \frac{1}{\tau_o'}- \frac{1}{\tau_0}$$


\section{Bimolecular Photochemistry}
\textbf{Rate equation} \\
Bimolecular: $\ce{S}^{*} + \ce{Q} \rightarrow \ce{P} \quad \nu_{\mathrm{bi}} = k_{\mathrm{bi}} [\ce{S}^{*}][\ce{Q}]$\\

\textbf{Examples} \\
Quenching: $\ce{A^*} + \ce{Q} \rightarrow \ce{A} + \ce{Q}$ \\
Electron transfer: $\ce{A^*} + \ce{B} \rightarrow \ce{A+} + \ce{B-}$ \\
Addition: $\ce{A^*} + \ce{B} \rightarrow \ce{AB}$ \\

Quenching:
\begin{equation*}
  \frac{\phi_{F}}{\phi_{F}'} = \frac{\tau_0}{\tau_0'} = \frac{k_0 + k_Q[Q]}{k_0} = 1 + \frac{k_Q}{k_0}[\ce{Q}]
\end{equation*}

Stern-Volmer Analysis:
$$\frac{\phi_F}{\phi_F'} = 1 + \tau_0 k_Q[\ce{Q}]$$

A plot of $\phi_F/\phi_F'$ against [\ce{Q}] will give a straight line of slope $\tau_0 k_Q$.
\section{Förster Resonance Energy Transfer}
1. Significant overlap of the absorption / emission spectrum of two molecules. \\
2. Both molecules are in close proximity. \\
FRET is efficient \textbf{non-radiative energy transfer} from the donor to the acceptor.
\vspace{\baselineskip}

FRET energy transfer efficiency:
$$\phi_{\ce{FRET}} = \frac{k_{\ce{FRET}}}{k_{\ce{FRET}}+k_0} = \frac{R_0^6}{R_0^6+R_{DA}^6}$$

where $R_{DA}$ is the distance between donor and acceptor, and 
$$k_{\ce{FRET}} = k_0 \left( \frac{R_0}{R_{DA}} \right)^6$$

FRET is extremely sensitive to small changes in distance.

\section{Electron Transfer}
