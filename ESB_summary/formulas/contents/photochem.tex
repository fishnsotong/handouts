\section{Photophysics}
\textbf{Franck-Condon Principle}. $\because$ nuclei are much more massive than electrons,
an electronic transition (electron transfer, absorption) takes place much faster than the nuclei can respond.
\subsection*{Fermi's Golden Rule}
$$k(\mathrm{s}^{-1}) = \frac{2\pi}{h}|S(f,i)|^2V^2$$

\textbf{Frank-Condon Factor}: The square of the overlap integral between the \textit{vibrational} wavefunctions.
$$|S(f,i)|^2 = \left( \int^{\infty}_{-\infty}\phi_i\phi_f \mathrm{d}x \right)^2$$

\subsection*{Light Absorption}
\textbf{Absorbance}: $$ A = -\log \frac{I}{I_0}$$ \\
$I$ is the intensity of transmitted light, and $I_0$ is the intensity of incident light

\vspace{\baselineskip}

\textbf{Beer Lambert Law}: $A = \epsilon cl$
\vspace{\baselineskip}

\textbf{Oscillator Strength} ($f$) is the empirical `probability' of absorption / emission of photon in electronic transitions.
\begin{equation*}
  f = 4 \times 10^{-9} \int_{\nu_1}^{\nu_2}\epsilon \mathrm{d}\nu \approx 4 \times 10^{-9} \epsilon_{\mathrm{max}} \Delta \nu_{1/2}
\end{equation*}

$f = 1$ is the `fully allowed' transition.

$\Delta \nu_{1/2}$ is the full-width at half maximum of the peak, converted to wavenumbers.
\vspace{\baselineskip}

\textbf{Transition Dipole}
\begin{equation*}
  \mathbf{\mu}_{fi} = -\mathrm{e} \int_{-\infty}^{\infty}\psi^{*}_{f} x \psi_{i} \mathrm{d}x
\end{equation*}

Transition dipole is related to oscillator strength: $|\nu_{fi}|^2 \propto f$
\vspace{\baselineskip}

\textbf{Selection Rules} 1. spin 2. symmetry \\
$\because$ conservation of angular momentum, $\Delta S =0 \therefore \mathrm{S_0} \not\rightarrow \mathrm{T_1} \wedge \mathrm{S_0} \not\rightarrow \mathrm{T_1}$

For centrosymmetric molecules, only $u \rightarrow g$ and  $g \rightarrow u$ transitions are allowed.

\subsection*{Light Emission}

Absorption:\\ $\ce{S} + h\nu_i \rightarrow \ce{S}^{*}\quad \nu_{\mathrm{abs}} = I_{\mathrm{abs}}$\\
Fluorescence:\\ $\ce{S}^{*} \rightarrow \ce{S} + h\nu_f \quad \nu_{\mathrm{F}} = k_{\mathrm{F}} [\ce{S}^{*}]$ \\
Internal conversion:\\ $\ce{S}^{*} \rightarrow \ce{S} \quad \nu_{\mathrm{IC}} = k_{\mathrm{IC}}[\ce{S}^{*}]$ \\
Intersystem crossing:\\ $\ce{S}^{*} \rightarrow \ce{T} \quad \nu_{\mathrm{ISC}} = k_{\mathrm{ISC}}[\ce{S}^{*}]$ \\


Quantum Yield:
$$\mathbf{\Phi}_F = \frac{k_R}{k_R + k_{IC} + k_{ISC}} = \frac{k_R}{k_0} = \frac{\tau_0}{\tau_R}$$

\section{Unimolecular Photochemistry}


\section{Bimolecular Photochemistry}
WHAT IS THIS?
\begin{equation*}
  \frac{\phi_{F}}{\phi_{F}'} = \frac{\tau_0}{\tau} = \frac{k_0 + k_Q[Q]}{k_0}
\end{equation*}
\section{Förster Resonance Energy Transfer}
A donor chromophore, initially in its electronic excited state, may transfer energy to an acceptor chromophore 
through non-radiative dipole-dipole coupling.
Dipole-dipole coupling:
$$E = \frac{1}{1 + (r/R_0)^6}$$

The efficiency of this energy transfer is inversely proportional to the sixth power of the distance between donor 
and acceptor, making FRET extremely sensitive to small changes in distance.

\section{Electron Transfer}
