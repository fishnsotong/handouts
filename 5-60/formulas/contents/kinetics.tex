\columnbreak
\section{Kinetics of Chemical Reactions}

\textbf{First-order reaction} \\
 $[\ce{A}]_t = [\ce{A}]_0 \exp (-kt)$
\vspace{\baselineskip}

\subsection*{Arrhenius Theory}

\begin{equation*}
    k = Ae^{-E_a/RT}
\end{equation*}

Small $E_a$ $\rightarrow$ weak $T$-dependence $\rightarrow$ fast \\
Large $E_a$ $\rightarrow$ strong $T$-dependence $\rightarrow$ slow \\
Is the thermal energy large or small compared to the activation energy?

\subsection*{Chain Reactions and Explosions}
Free radical chain length: rate of product formation / rate of radical formation. By making the SSA, the expression for kinetic chain length $v$ is
\begin{equation*}
    v = \frac{k_p [\ce{M.}][\ce{M}]}{2k_t[\ce{M.}]^2} = \frac{k_p[\ce{M.}]}{2k_t[\ce{M.}]}
\end{equation*}

Explosions happen when concentration of reactive intermediate becomes large, and the
steady-state approximation fails.

\subsection*{Catalysis}
Catalysts only increase reaction rate. $K_{\mathrm{eq}}$, $\Delta H$, $\Delta S$, $\Delta G$ unchanged.

\subsection*{Michaelis-Menten}

\subsection*{Chemical Oscillations}
Occur with two autocatalytic steps.