\pagebreak

\section{Electrochemistry}

\subsection*{General}
$V = RI$ \\
$P = VI = I^2R = V^2/R$ \\
$R = \rho L / A$

Capacitance ($C$). A dielectric material can be polarised: dipoles can align in an electric field
but charges cannot move.
$$Q = -C \Delta E = -\left(\frac{\epsilon A}{L}\right) \Delta E$$\underline{Nomenclature}

\textit{Cathode} - reduction occurs ($-$ve current)\\
\textit{Anode} - oxidation occurs  (+ve current)\\
\textit{Electrochemical eqm} - no net current flow due to very high $R$ or external applied $V$ ($w = 0$, $E = E^\standardstate$, $I = 0$) - reversible\\
\textit{Galvanic cell} - work done on the system, ($w < 0$, $E < E^\standardstate$, $I > 0$) - oxidation\\
\textit{Electrolytic cell} - work done on the cell, ($w > 0$, $E > E^\standardstate$, $I < 0$) - reduction\\
\textbf{Half-cell reactions.} By convention, they are written as reductions. $ E >> 0$
means reaction is strongly oxidising, removing electrons from others (being reduced).
\textbf{Cell potential ($E_{\mathrm{cell}}$).} As shown, with balancing electrons
\begin{equation*}
    E_{\mathrm{cell}} = E_{\mathrm{cathode}} - E_{\mathrm{anode}}
\end{equation*}
\subsection*{Mass Transport}

Gas phase: coefficient? \\
Solution phase: coefficient?

\textbf{Fick's First Law.}

\textbf{Fick's Second Law.}

\textbf{Langmuir adsorption isotherm}

\textbf{BET adsorption isotherm}
Adsorption is not limited to monolayer.

The distance travelled in one dimension by diffusion is $x = (2D_0t)^{1/2}$.

\textbf{Double layer.} Layer of well-ordered

\textbf{Sherwood number.} Does advection or diffusion dominate mass transport in the system?

\textbf{ionic strength}

\textbf{diffusion layer}

\subsection*{Thermodynamics}
\textbf{Relationship between $\Delta G$ \& cell emf.}
\begin{equation*}
    \Delta G = -nFE
\end{equation*}
\textbf{Nernst Equation.} Allows us to relate cell potential to the actual concentration of species
present and standard reduction potential ($E^\standardstate$).
\begin{equation*}
    E = E^\standardstate - \frac{RT}{nF} \ln Q = E^\standardstate - \frac{RT}{nF} \ln\left(\frac{\alpha_B}{\alpha_A}\right)
\end{equation*}

Nernst equation holds at electrochemical equilibrium.



\textbf{Potentiostatic measurements.} Recalling $\Delta G = \Delta H - T \Delta S$, 
$ \therefore - nFE = \Delta H - T \Delta S$. We can use electrochemical measurements to determine
thermodynamic properties.
\begin{equation*}
    E = \dv{E}{T}T - \frac{\Delta H}{nF}
\end{equation*} The slope of a graph of potential ($E$) against temperature ($T$)
is $\Delta S / nF$ and the intercept is $- \Delta H / nF$.

\subsection*{Electron transfer kinetics}

\textbf{Butler-Volmer equation}

\textbf{Eyring-Polanyi equation}

\textbf{Tafel plot}

\subsection*{Processes at solid electrodes}


\subsection*{Final remarks}
You can control potential and measure current (voltammetry) or you can control current
and measure voltage (galvanometry) but you cannot control both current and voltage
simultaneously!

