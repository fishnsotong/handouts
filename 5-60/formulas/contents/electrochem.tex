\section{Electrochemistry}

\subsection*{General}
$V = RI$ \\
$P = VI = I^2R = V^2/R$ \\
$R = \rho L / A$

\textbf{Capacitance} $C$ is defined by $Q=C\phi$, gives a measure of how much charge a conductor can hold
for a given potential.

For a sphere and parallel-plate capacitor, their capacitances are:
\begin{equation*}
    C_{\textrm{sphere}} = 4 \pi \epsilon_0 r \quad C_{\textrm{plates}} \frac{\epsilon_0 A}{L}
\end{equation*}

The energy in a capacitor is
\begin{equation*}
    U = \frac{1}{2}C\phi^2 = \frac{Q^2}{2C} = \frac{1}{2}Q\phi
\end{equation*}
Between two parallel conducting plates, $\Delta \phi = Ed$. The energy density (energy stored per unit volume) is:
\begin{equation*}
    u_d = \frac{1}{2}\epsilon_0 E^2.
\end{equation*}

A dielectric material can be polarised: dipoles can align in an electric field
but charges cannot move.

\columnbreak
\underline{Nomenclature}

\textit{Cathode} - reduction occurs ($-$ve current)\\
\textit{Anode} - oxidation occurs  (+ve current)\\
\textit{Electrochemical eqm} - no net current flow due to very high $R$ or external applied $V$ ($w = 0$, $E = E^\standardstate$, $I = 0$) - reversible\\
\textit{Galvanic cell} - work done on the system, ($w < 0$, $E < E^\standardstate$, $I > 0$) - oxidation\\
\textit{Electrolytic cell} - work done on the cell, ($w > 0$, $E > E^\standardstate$, $I < 0$) - reduction\\
\textbf{Half-cell reactions.} By convention, they are written as reductions. $ E >> 0$
means reaction is strongly oxidising, removing electrons from others (being reduced).
\textbf{Cell potential ($E_{\mathrm{cell}}$).} As shown, with balancing electrons
\begin{equation*}
    E_{\mathrm{cell}} = E_{\mathrm{cathode}} - E_{\mathrm{anode}}
\end{equation*}
\subsection*{Mass Transport}

Gas phase: coefficient? \\
Solution phase: coefficient?

\textbf{Fick's First Law.}

\textbf{Fick's Second Law.}

\textbf{Langmuir adsorption isotherm}

\textbf{BET adsorption isotherm}
Adsorption is not limited to monolayer.

The distance travelled in one dimension by diffusion is $x = (2D_0t)^{1/2}$.

\textbf{Double layer.} Layer of well-ordered

The thickness of the electrical double layer can be estimated by Debye-H\"uckel
\begin{equation*}
    \lambda_D = \sqrt{\frac{\epsilon_r \epsilon_0RT}{2F^2Im^\standardstate\rho}} = \sqrt{\frac{\epsilon_r \epsilon_0 k_B T}{2N_A^2q_e^2Im^\standardstate\rho}}
\end{equation*}

\textbf{Sherwood number.} Does advection or diffusion dominate mass transport in the system?

\textbf{ionic strength}

\textbf{diffusion layer}

\subsection*{Thermodynamics}
\textbf{Relationship between $\Delta G$ \& cell emf.}
\begin{equation*}
    \Delta G = -nFE
\end{equation*}
\textbf{Nernst Equation.} Allows us to relate cell potential to the actual concentration of species
present and standard reduction potential ($E^\standardstate$).
\begin{equation*}
    E = E^\standardstate - \frac{RT}{nF} \ln Q = E^\standardstate - \frac{RT}{nF} \ln\left(\frac{\alpha_B}{\alpha_A}\right)
\end{equation*}

Nernst equation holds at electrochemical equilibrium.

\begin{equation*}
    Q = \prod_j \left(\frac{a_j}{a^\standardstate}\right)^{v_j}
\end{equation*}
where $v_j$ is the stoichiometric coefficient

\textbf{Potentiostatic measurements.} Recalling $\Delta G = \Delta H - T \Delta S$, 
$ \therefore - nFE = \Delta H - T \Delta S$. We can use electrochemical measurements to determine
thermodynamic properties.
\begin{equation*}
    E = \dv{E}{T}T - \frac{\Delta H}{nF}
\end{equation*} The slope of a graph of potential ($E$) against temperature ($T$)
is $\Delta S / nF$ and the intercept is $- \Delta H / nF$.

\subsection*{Electron transfer kinetics}

\textbf{Butler-Volmer equation}
\begin{equation*}
    j = j_0 \left( \exp \frac{(1 - \beta)qF\eta}{RT} - \exp \frac{-\beta q F \eta}{RT} \right)
\end{equation*}

\textbf{Eyring-Polanyi equation}

where $B = k_BT/h$.

\textbf{Tafel plot}

\subsection*{Processes at solid electrodes}
\begin{equation*}
    k_{\textrm{MT}} = \frac{D_O}{\delta_D} \quad j_{\textrm{MT}} = -qF k_{\textrm{MT}}a_{o^{Z+}}c^o
\end{equation*}
\begin{equation*}
    z = \exp \left( \frac{qF(E - E_{1/2})}{RT} \right)
\end{equation*}

\begin{equation*}
    \frac{1}{j_{\textrm{mix}}} = \frac{1}{j_{\textrm{kinetic}}} + \frac{1}{j_{\textrm{MT}}} 
\end{equation*}
\begin{equation*}
    j_{\textrm{kinetic}} = \frac{j_{\textrm{MT}}j_{\textrm{mix}}}{j_{\textrm{MT}} - j_{\textrm{mix}}}
\end{equation*}

\subsection*{Final remarks}
You can control potential and measure current (voltammetry) or you can control current
and measure voltage (galvanometry) but you cannot control both current and voltage
simultaneously!

