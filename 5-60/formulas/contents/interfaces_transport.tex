\section{Interfaces}

\subsection*{Fluid interfaces} % these include liquid-vapour interfaces, right?
$\differential w = f \differential h = 2 \gamma l \differential h$
\begin{equation*}
    \Gamma_i = \frac{n_i(\sigma)}{\sigma}
\end{equation*}

\subsection*{Solid-liquid-vapour interfaces}

\subsection*{Langmuir adsorption isotherm}
\begin{equation*}
    \pdv{\ln(KP^\standardstate)}{(1/T)} = - \left(\frac{\Delta_{\textrm{ads}}H}{R}\right)
\end{equation*}

\subsection*{BET adsorption isotherm}
\begin{equation*}
    Ka_x \exp(\frac{(\phi_M - \phi_S)F}{RT}) = \frac{\theta}{1 - \theta}
\end{equation*}

\section{Transport: molecules in motion}

\subsection*{Advection and mass transport}

When considering mass transport, we may have to take into account

\subsection*{Migration}
Migration is motion due to an electric field due to Coulombic force.

\subsection*{Diffusion} % is there a difference between migration and diffusion? maybe should just be diffusion
Diffusion is motion due to a difference in chemical potential.

\textbf{Diffusion coefficient.} $$D = \frac{k_B T}{6 \pi \eta a}$$

where $a$ is the hydrodynamic radius, and $\eta$ is the viscosity coefficient.

Ions reach a terminal velocity when the electrical force applied on them is balanced
by drag due to solvent viscosity.
\vspace{\baselineskip}

\textbf{Drift speed of ions in electric field.} The mobility $u$ of an ion decreases with increasing solution viscosity and ion size.
$$s = uE = \frac{ze}{6 \pi \eta a} E$$

The diffusion coefficient $D$ can be used to find $u$.
\begin{equation*}
    u = \frac{zeD}{k_B T}
\end{equation*}

\subsection*{Thermophoresis}
Thermophoresis is particle drift induced by thermal gradients. A particle moving in a temperature gradient
$\grad T$ will have a steady thermophoretic drift velocity $v_T$
\begin{equation*}
    \begin{aligned}
        v_T &=-D_T \grad T \\
        &= -D_T (T_2 - T_1)
    \end{aligned}
\end{equation*}
$D_T$ is the thermal diffusion coefficient.
By \textit{Fick's first law}, we see that mass flux of thermophoresis counteracts diffusion
\begin{equation*}
    J = -D \dv{c}{x} - c D_T \dv{T}{x}
\end{equation*}
Therefore when $J = 0$,
\begin{equation*}
    \begin{aligned}
        \frac{c(T_2)}{c(T_1)} &= e^{-\frac{D_T}{D}(T_2 - T_1)} \\
        &= e^{-S_T(T_2 - T_1)}
    \end{aligned}
\end{equation*}
Soret coefficient $S_T$ is a measure of the strength of the thermophoretic molecule flow compared with ordinary diffusion.
\begin{equation*}
    S_T = \frac{D_T}{D}
\end{equation*}
$S_T > 0$ for a thermophobic particle and $S_T < 0$ for a thermophilic particle.
\begin{equation*}
    S_T = \frac{A}{kT} \left( - \Delta_{\textrm{hyd}}S(T) + \frac{\beta \sigma^2_{\textrm{eff}}}{4 \epsilon \epsilon_0T}\lambda_{\textrm{DH}}\right)
\end{equation*}
\subsection*{Magnetophoresis}
Magnetophoresis is motion of particles due to an magnetic field. According to their magnetic susceptibility $\chi$,
materials are classified as: \\
ferromagnetic ($\chi >> 0$) \\
paramagnetic ($\chi > 0$) \\
diamagnetic ($\chi < 0$).

The average velocity when $F_m = F_{\textrm{drag}}$ is:
\begin{equation*}
    v_{\textrm{mag}} = \frac{F_m}{6 \pi \eta a}
\end{equation*}
$\eta$ is  viscosity and $a$ is hydrodynamic size.