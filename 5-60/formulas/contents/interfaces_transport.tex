\section{Interfaces}

\subsection*{Fluid interfaces} 
We need to do work to increase the area of an interface:
$\differential w = f \differential h = 2 \gamma l \differential h$

The magnitude of the surface tension $\gamma$ is proportional to the cohesive energy $u$, where $\sigma_m$ is molecular area
\begin{equation*}
    y \approx \frac{u}{2 \sigma_m}
\end{equation*}

$G$ quantifies the work associated to changes in the interfacial area at constant $T$ and $p$.
\begin{equation*}
    \differential G = -S \differential T + V \differential p  + \gamma \differential \sigma
\end{equation*}
which can be expressed as a partial derivative
\begin{equation*}
    \gamma = \left(\pdv{G}{\sigma}\right)_{T,p}
\end{equation*}
Interfacial area is minimised at equilibrium, forming curved surfaces.

\textbf{Young-Laplace equation} gives us the pressure difference
\begin{equation*}
    \Delta P = p_i - p_o = \frac{2 \gamma}{r}
\end{equation*}

\textbf{Surface excess} informs us about the number of molecules per unit area
\begin{equation*}
    \Gamma_i = \frac{n_i(\sigma)}{\sigma}
\end{equation*}

$\Gamma_i > 0$: positive adsorption of component at the surface - surface active

$\Gamma_i < 0$: depletion of component i from the surface

Adsorption of molecules modifies the surface/interfacial tension according to the \textbf{Gibbs isotherm}
\begin{equation*}
    \differential \gamma = - \sum_{i} \Gamma_i \differential \mu_i
\end{equation*}

\subsection*{Solid-liquid-vapour interfaces}

\subsection*{Langmuir adsorption isotherm}
\begin{equation*}
    \pdv{\ln(KP^\standardstate)}{(1/T)} = - \left(\frac{\Delta_{\textrm{ads}}H}{R}\right)
\end{equation*}

\subsection*{BET adsorption isotherm}
Adsorption is not limited to monolayer.

This is the electrosorption isotherm.
\begin{equation*}
    Ka_x \exp(\frac{(\phi_M - \phi_S)F}{RT}) = \frac{\theta}{1 - \theta}
\end{equation*}

\section{Transport: molecules in motion}

\subsection*{Advection and mass transport}
Advection is bulk motion of a fluid due to stirring and flow.

\subsection*{Migration}
Migration is motion due to an electric field by Coulombic force.

conductivity $\sigma$ and resistivity $\rho$ of a material are related to current density $\mathbf{J}$, which is
amount of current flowing per unit cross-sectional area $I = JA$.
\begin{equation*}
    \sigma = \frac{J}{E} \quad \rho = \frac{1}{\sigma} = \frac{E}{J}
\end{equation*}

Conductivity $\kappa$ to compare electrolytes $\kappa = 1/\rho = L/RA$. Molar conductivity to consider concentration effects
\begin{equation*}
    \Lambda_m = \frac{\kappa}{c}
\end{equation*}

$\Lambda_m$ of strong electrolytes decreases (a little)as concentration goes up. This is due to ion-ion and ion-solvent interactions.

\textbf{Law of independent migration of ions} considers $\Lambda_m^o$, at infinite dilution where ions do not interact with each other.
\begin{equation*}
    \Lambda_m^o = n_{+} \lambda_{+}^o + n_{-} \lambda_{-}^o
\end{equation*}

\textbf{Factors affecting ion speed}

(1) Driving force $F = zeE$

(2) Frictional force $F = 6 \pi \eta a s$

And we get an expression for the \textit{drift speed of an ion} in an \textbf{E} field:
\begin{equation*}
    s = \frac{zeE}{6 \pi \eta a}
\end{equation*}
where $a$ is the hydrodynamic radius, and $\eta$ is the viscosity coefficient.

Mobility $u$ is defined by $s = uE$.

\textbf{Drift speed of ions in electric field.} The mobility $u$ of an ion decreases with increasing solution viscosity and ion size.
$$s = uE = \frac{ze}{6 \pi \eta a} E$$

\subsection*{Diffusion} % is there a difference between migration and diffusion? maybe should just be diffusion
Diffusion is motion due to a difference in chemical potential.

\textbf{Diffusion coefficient.} $$D = \frac{k_B T}{6 \pi \eta a}$$

D doesn’t depend on charge (\textit{diffusion is not driven by charge effects}), D is inversely proportional to viscosity through the opposing effect of friction.


Ions reach a terminal velocity when the electrical force applied on them is balanced
by drag due to solvent viscosity.
\vspace{\baselineskip}

The diffusion coefficient $D$ can be used to find mobility $u$.
\begin{equation*}
    u = \frac{zeD}{k_B T}
\end{equation*}
by relating $s = uE$ and $s = DF /RT$. We can link conductivity parameters to $u$:
\begin{equation*}
    \lambda_m = |z|uF = \frac{z^2DF}{RT}
\end{equation*}
Limiting molar conductivity: 
\begin{equation*}
    \begin{aligned}
        \Lambda_m^o &= (|z|_{+}u_{+}n_{+} + |z|_{-}u_{-}n_{-})F \\
        &= (n_{+}z^2_{+}D_{+} + n_{-}z^2_{-}D_{-})\frac{F^2}{RT}
    \end{aligned}
\end{equation*}
Can use conductivity to measure ionic diffusion coefficients.

\textbf{Fick's First Law}
Flux $J$ is given by the product of the diffusion coefficient and concentration gradient
\begin{equation*}
    J = -D \derivative{c}{x}
\end{equation*}

\textbf{Fick's Second Law}
\begin{equation*}
    \pdv{c}{t} = D \pdv[2]{c}{x}
\end{equation*}

\subsection*{Thermophoresis}
Thermophoresis is particle drift induced by thermal gradients. A particle moving in a temperature gradient
$\grad T$ will have a steady thermophoretic drift velocity $v_T$
\begin{equation*}
    \begin{aligned}
        v_T &=-D_T \grad T \\
        &= -D_T (T_2 - T_1)
    \end{aligned}
\end{equation*}
$D_T$ is the thermal diffusion coefficient.
By \textit{Fick's first law}, we see that mass flux of thermophoresis counteracts diffusion
\begin{equation*}
    J = -D \dv{c}{x} - c D_T \dv{T}{x}
\end{equation*}
Therefore when $J = 0$,
\begin{equation*}
    \begin{aligned}
        \frac{c(T_2)}{c(T_1)} &= e^{-\frac{D_T}{D}(T_2 - T_1)} \\
        &= e^{-S_T(T_2 - T_1)}
    \end{aligned}
\end{equation*}
Soret coefficient $S_T$ is a measure of the strength of the thermophoretic molecule flow compared with ordinary diffusion.
\begin{equation*}
    S_T = \frac{D_T}{D}
\end{equation*}
$S_T > 0$ for a thermophobic particle and $S_T < 0$ for a thermophilic particle.
\begin{equation*}
    S_T = \frac{A}{kT} \left( - \Delta_{\textrm{hyd}}S(T) + \frac{\beta \sigma^2_{\textrm{eff}}}{4 \epsilon \epsilon_0T}\lambda_{\textrm{DH}}\right)
\end{equation*}
\subsection*{Magnetophoresis}
Magnetophoresis is motion of particles due to an magnetic field. According to their magnetic susceptibility $\chi$,
materials are classified as: \\
ferromagnetic ($\chi >> 0$) \\
paramagnetic ($\chi > 0$) \\
diamagnetic ($\chi < 0$).

The average velocity when $F_m = F_{\textrm{drag}}$ is:
\begin{equation*}
    v_{\textrm{mag}} = \frac{F_m}{6 \pi \eta a}
\end{equation*}
$\eta$ is  viscosity and $a$ is hydrodynamic size.
