\pagebreak
\section{Statistical Mechanics}
The Boltzmann constant ($k$ or $k_B$) is the proportionality factor which relates the average relative thermal energy 
of gas particles with the temperature of the gas.
$k = 1.38 \times 10^{-23} \mathrm{J} \mathrm{K}^{-1}$, $kN = Rn$

\subsection*{Boltzmann Equation}
$S = k \ln \Omega$, where $\Omega$ is the multiplicity; number of possible states.
\begin{equation*}
    S = -k \sum_{i = 1}^{t}p_i \ln p_i
\end{equation*}

\subsection*{Stirling's Approximation}
\begin{equation*}
    n! \approx \left(\frac{n}{e}\right)^n
\end{equation*}

$\ln n! = n \ln n - n$, $\ln n^x = x \ln n$.

\subsection*{Boltzmann Distribution Law}

\subsection*{Partition Function}
\begin{equation*}
    Q_{\mathrm{sys}} = \prod_{i} q_i
\end{equation*}

For $N$ particles which are \\
independent and distinguishable: $$Q = q^N$$independent and indistinguishable: $$Q = q^N/N!$$
$\beta = 1/kT$, $E = U/N$, so $$U = kT^2\left(\frac{\partial \ln Q}{\partial T}\right)$$

\subsection*{Absolute Entropy}