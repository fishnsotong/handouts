\section{Laws of Thermodynamics}
These laws are completely empirical.

\begin{enumerate}
    \item U is conserved. $\delta w + \delta q = \mathrm{d}U$
    \item S is defined (direction of time). $ \Delta S = \int \mathrm{d}q_{\mathrm{rev}} / T$
    \item Absolute scale. $S \rightarrow 0$ as $T \rightarrow 0$ for a pure crystal
\end{enumerate}

\section{Properties of Gases}
\subsection*{Equations of state}
\textbf{Ideal Gas Law}.
\begin{equation*}
    pV = nRT
\end{equation*}
\textbf{van der Waals equation.} Equation of state for a real gas.
$$\left(p + \frac{a}{\bar{V}^2}\right) \left(\bar{V} - b\right) = RT$$ 

$a$ representing molecular attractions and $b$ representing molecular repulsions.

\section{Energetics and Equilibria}
\subsection*{Useful conventions}
If heat enters system, $q > 0$\\
system does work on surroundings, $w < 0$\\
surroundings do work on system, $w > 0$

\subsection*{Internal energy \& enthalphy}
\textbf{Enthalpy} is heat transfer at constant $p$, which is useful in laboratory conditions.
\begin{equation*}
    H = U + pV
\end{equation*}

\textbf{Entropy} of an isolated system increases in a spontaneous process. Statistically,
\begin{equation*}
    S = k_B \ln \Omega.
\end{equation*}
Thermodynamically,
\begin{equation*}
    \differential S \geq \frac{\differential q_{\textrm{rev}}}{T}
\end{equation*}

\subsection*{Gibbs Energy}
\begin{equation*}
    G = H - TS
\end{equation*}
In a spontaneous process $G$ of a system \textit{decreases} and reaches a minimum at equilibrium.
\begin{equation*}
    \begin{aligned}
        \differential G &= -S \differential T \\
        \differential G &= -V \differential p
    \end{aligned}
\end{equation*}

$G$ varies with $p$ as shown:
\begin{equation*}
    G_m(p_2) = G_m(p_1) + RT \ln \left(\frac{p_2}{p_1}\right)
\end{equation*}

\columnbreak
\subsection*{Chemical Equilibrium}
\begin{equation*}
    \Delta_{\mathrm{r}}G^\standardstate = -RT \ln K
\end{equation*}
$K$ is the thermodynamic equilibrium constant: independent of pressure.

\textbf{Van't Hoff Equation.} Relates $K$ with $T$.

\begin{equation*}
    \dv{\ln K}{T} = \frac{\Delta_{\mathrm{r}}H^\standardstate}{RT^2}
\end{equation*}
The integral form is more explicit, if we can assume that $\Delta_{\mathrm{r}}H^\standardstate$ does not change from $T_1$ to $T_2$.
\begin{equation*}
    \ln K (T_2) - \ln K (T_1) = \frac{-\Delta_{\mathrm{r}}H^\standardstate}{R} \left(\frac{1}{T_2} - \frac{1}{T_1}\right)
\end{equation*}
Importantly, a plot of $\ln K$ against $1/T$ is a straight line with slope $-\Delta_{\mathrm{r}}H^\standardstate / R$.

\section{Phase equilibria: pure substances}

\textbf{Phase rule: degrees of freedom} is the number of intensive variables that can be varied independently
\begin{equation*}
    F = C - P + 2
\end{equation*}
where $P$ is number of phases and $C$ is the number of components. $P = 4$ cannot exist for a one-component system.

\subsection*{Liquid-vapour coexistence}
At the \textbf{critical point}, interaction strength $\epsilon$ is similar to the thermal energy.
\begin{equation*}
    \frac{k_bT_c}{\epsilon} \approx 1
\end{equation*}
Liquids are formed below the critical point, with intermolecular interactions of sufficient strength.

\subsection*{Chemical potential}
For pure substances, \textit{chemical potential} is molar Gibbs energy $\mu = G_m$. In mixtures, it is the \textit{partial}
molar Gibbs energy.
\begin{equation*}
    \mu_i = \left(\pdv{G}{n_i}\right)_{p,T}
\end{equation*}
At equilibrium, the chemical potential of a substance is the same throughout every phase present in the system. This
gives us the \textbf{coexistence condition}:
\begin{equation*}
    \mu(\alpha;p,T) = \mu(\beta;p,T)
\end{equation*}

\subsection*{Clausius-Clapeyron equation}
The Clapeyron equation is an exact expression for the slope of a phase boundary.
Thermodynamic data can be used to predict the slope of phase diagrams $\dv*{p}{T}$.
\begin{equation*}
    \dv{p}{T} = \frac{\Delta_{\textrm{trs}}S}{\Delta_{\textrm{trs}}V}
\end{equation*}
This has numerous applications, such as predicting how freezing and boiling points change when pressure is applied:
\begin{equation*}
    \dv{T}{p} = \frac{\Delta_{\textrm{trs}}V}{\Delta_{\textrm{trs}}S}
\end{equation*}
At the \textit{solid-liquid boundary},
\begin{equation*}
    \dv{p}{T} = \frac{\Delta_{\textrm{fus}}H}{T\Delta_{\textrm{fus}}V}
\end{equation*}
Usually, $\dv*{p}{T} > 0$, as the pressure is raised, the melting temperature rises.

At the \textit{liquid-vapour boundary},
\begin{equation*}
    \dv{p}{T} = \frac{\Delta_{\textrm{vap}}H}{T\Delta_{\textrm{vap}}V}
\end{equation*}
$\Delta_{\textrm{vap}}V$ is large, boiling $T$ is more responsive to pressure than freezing $T$.

Sufficiently far from the critical point we assume the vapour is ideal, $V_m = RT / p$, giving the Clausius-Clapeyron equation, which tells us how vapour pressure varies with $T$.
\begin{equation*}
    \frac{\mathrm{d}\ln p}{\mathrm{d}T} \approx \frac{\Delta H_{\textrm{vap}}}{RT^2}
\end{equation*}
Integrating between $(p_1, p_2)$ and $(T_1, T_2),$
\begin{equation*}
    \begin{aligned}
        \ln\left(\frac{p_2}{p_1}\right) &= \frac{\Delta H_{\textrm{vap}}}{R}\left(\frac{1}{T_1} - \frac{1}{T_2}\right) \\
        p_2 &= p_1 \exp \left(- \frac{\Delta H_{\textrm{vap}}}{R}\left[\frac{1}{T_1} - \frac{1}{T_2}\right]\right)
    \end{aligned}
\end{equation*}

\section{Phase equilibria: simple mixtures}

\subsection*{Fundamental equation}
Because the chemical potentials depend on composition (and $p$ and $T$), the Gibbs energy of a mixture may change when these variables change.
\begin{equation*}
    \differential G = -S \differential T + V \differential p + \mu_A \differential n_A + \mu_B \differential n_B
\end{equation*}
For a multi-component mixture,
\begin{equation*}
    \differential G = -S \differential T + V \differential p + \sum_i \mu_i \differential n_i
\end{equation*}

$\therefore$ total free energy of binary mixture at constant T and p is $G = \mu_A n_A + \mu_B n_B$.
\subsection*{Gibbs-Duhem equation}
At constant $p$ and $T$, chemical potentials of the components in a mixture cannot change independently.
\begin{equation*}
    n_A \differential \mu_A + n_B \differential \mu_B = 0.
\end{equation*}
For a multi-component mixture,
\begin{equation*}
    \sum_i n_i \differential \mu_i = 0
\end{equation*}

\subsection*{Mixtures of ideal gases}
Variation of chemical potential with $p$:
\begin{equation*}
    \mu = \mu^\standardstate + RT \ln \left(\frac{p}{p^\standardstate}\right)
\end{equation*}
$\mu_\standardstate$ is the chemical potential of pure gas at 1 bar. Which becomes $ \mu = \mu^\standardstate + RT \ln p$. 
Mixing of two ideal gases is \textit{spontaneous}, driven entirely by entropy $\because$ there are no intermolecular interactions.
\vspace{\baselineskip}

\textbf{Thermodynamic mixing functions.}
\begin{equation*}
    \begin{aligned}
    \Delta_{\textrm{mix}}G &= nRT(x_A \ln x_A + x_B \ln x_B) \\
    \Delta_{\textrm{mix}}S &= - \left( \pdv{\Delta_{\textrm{mix}}G}{T}\right)_p \\
    &= -nR(x_A \ln x_A + x_B \ln x_B)
    \end{aligned}
\end{equation*}

\subsection*{Raoult's law}
This equation shows that we can obtain $\mu$ of a liquid by measuring
change in its vapour pressure, givenvapour pressure of the pure liquid is $p_A^*$
\begin{equation*}
    \mu_A = \mu_A^* + RT \ln \left(\frac{p_A}{p_A^*}\right)
\end{equation*}
We assume the vapour pressure in ideal solutions is given by \textbf{Raoult's law}.
\begin{equation*}
    p_A = x_A p_A^*
\end{equation*}
It follows from the two equations above that chemical potential in ideal solutions is given by
\begin{equation*}
    \mu_A = \mu_A^* + RT \ln x_A
\end{equation*}

\subsection*{Henry's law}
Many mixtures significantly deviate from Raoult's law as they are non-ideal (with intermolecular interactions).
\begin{equation*}
    p_B = x_B K_B
\end{equation*}

Positive deviations: \\
A-B interactions $>$ A-A \& A-B interactions

Negative deviations: \\
A-B interactions $<$ A-A \& A-B interactions

\subsection*{Colligative properties}
1. Vapour pressure lowering \\
$\Delta p_A = p_A - p^{*}_{A} = -x_B p^*_A$ \\
2. Boiling point elevation \\
3. Freezing point depression \\
4. Osmotic pressure \\
$\Pi = [B]RT$

\subsection*{Phase coexistence of mixtures}
\begin{equation*}
    \begin{aligned}
        p & = p_A + p_B \\
        &= x_Ap_A^* + x_Ap_A^* \\
        &= p^*_B + (p_A^* - p_B^*)x_A
    \end{aligned}
\end{equation*}

\subsection*{Debye H\"uckel theory}



 