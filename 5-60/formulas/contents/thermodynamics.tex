\section{Laws of Thermodynamics}
These laws are completely empirical.

\begin{enumerate}
    \item U is conserved. $\delta w + \delta q = \mathrm{d}U$
    \item S is defined (direction of time). $ \Delta S = \int \mathrm{d}q_{\mathrm{rev}} / T$
    \item Absolute scale. $S \rightarrow 0$ as $T \rightarrow 0$ for a pure crystal
\end{enumerate}

\section{Properties of Gases}
Ideal Gas Law $$pV = nRT$$

\textbf{van der Waals.} Takes into account intermolecular interactions, and that gases take up space.
$$\left(p + \frac{a}{\bar{V}^2}\right) \left(\bar{V} - b\right) = RT$$ 

If heat enters system, $q > 0$\\
system does work on surroundings, $w < 0$\\
surroundings do work on system, $w > 0$

\section{Energetics and Equilibria}

\subsection*{Gibbs Energy}
\begin{equation*}
    G = H - TS
\end{equation*}
In a spontaneous process $G$ of a system \textit{decreases} and reaches a minimum at equilibrium.

\subsection*{Chemical Equilibrium}
\begin{equation*}
    \Delta_{\mathrm{r}}G^\standardstate = -RT \ln K
\end{equation*}

\textbf{Van't Hoff Equation.}

\begin{equation*}
    \odv{\ln K}{T} = \frac{\Delta_{\mathrm{r}}H^\standardstate}{RT^2}
\end{equation*}

The integral form is more explicit, if we can assume that $\Delta_{\mathrm{r}}H^\standardstate$ does not change from $T_1$ to $T_2$.
\begin{equation*}
    \ln K (T_2) - \ln K (T_1) = \frac{-\Delta_{\mathrm{r}}H^\standardstate}{R} \left(\frac{1}{T_2} - \frac{1}{T_1}\right)
\end{equation*}
Importantly, a plot of $\ln K$ against $1/T$ is a straight line with slope $-\Delta_{\mathrm{r}}H^\standardstate / R$.

\section{Phase Equilibria and Mixtures}

\subsection*{Clausius-Clapeyron}
What is this shit?
\begin{equation*}
    \frac{\mathrm{d}\ln p}{\mathrm{d}T} \approx \frac{\Delta H}{RT^2}
\end{equation*}
which can be simplified into (why? what approximations am I making?)
\begin{equation*}
    \ln\left(\frac{p_2}{p_1}\right) = \frac{\Delta H}{R}\left(\frac{1}{T_1} - \frac{1}{T_2}\right)
\end{equation*}

\begin{equation*}
    \Delta V_{\mathrm{fus}} = \frac{1}{\rho_l} - \frac{1}{\rho_s}
\end{equation*}

\subsection*{Raoult's Law}

\subsection*{Henry's Law}

\subsection*{Colligative Properties}
1. Vapour pressure lowering \\
$\Delta p_A = p_A - p^{*}_{A} = -x_B p^*_A$ \\
2. Boiling point elevation \\
3. Freezing point depression \\
4. Osmotic pressure \\

\subsection*{Mass Transport} %TODO: not really sure what to name this, but should give it a good name.

\textbf{Diffusion coefficient.} $$D = \frac{k_B T}{6 \pi \eta a}$$

where $a$ is the hydrodynamic radius, and $\eta$ is the viscosity coefficient.

Ions reach a terminal velocity when the electrical force applied on them is balanced
by drag due to solvent viscosity.
\vspace{\baselineskip}

\textbf{Drift speed of ions in electric field.} The mobility $u$ of an ion decreases with increasing solution viscosity and ion size. $$s = uE = \frac{ze}{6 \pi \eta a} E$$

The diffusion coefficient $D$ can be used to find $u$.
\begin{equation*}
    u = \frac{zeD}{k_B T}
\end{equation*}

 