\documentclass[a4paper]{tufte-handout}

\title{23. Colligative Properties\thanks{Wayne~W.~Z. Yeo}}

\author[MIT 5.60]{\textnormal{MIT 5.60} Thermodynamics and Kinetics\thanks{Course Instructors: Keith A. Nelson, Moungi Bawendi}}

%\date{28 March 2010} % without \date command, current date is supplied

%\geometry{showframe} % display margins for debugging page layout

\usepackage{graphicx} % allow embedded images
  \setkeys{Gin}{width=\linewidth,totalheight=\textheight,keepaspectratio}
  \graphicspath{{graphics/}} % set of paths to search for images
\usepackage{amsmath,amsthm}  % extended mathematics
\usepackage{physics,siunitx}
\usepackage[version=4]{mhchem} \usepackage{chemmacros}
\usepackage{booktabs} % book-quality tables
\usepackage{units}    % non-stacked fractions and better unit spacing
\usepackage{multicol} % multiple column layout facilities
\usepackage{lipsum}   % filler text
\usepackage{fancyvrb} % extended verbatim environments
  \fvset{fontsize=\normalsize}% default font size for fancy-verbatim environments

% Standardize command font styles and environments
\newcommand{\doccmd}[1]{\texttt{\textbackslash#1}}% command name -- adds backslash automatically
\newcommand{\docopt}[1]{\ensuremath{\langle}\textrm{\textit{#1}}\ensuremath{\rangle}}% optional command argument
\newcommand{\docarg}[1]{\textrm{\textit{#1}}}% (required) command argument
\newcommand{\docenv}[1]{\textsf{#1}}% environment name
\newcommand{\docpkg}[1]{\texttt{#1}}% package name
\newcommand{\doccls}[1]{\texttt{#1}}% document class name
\newcommand{\docclsopt}[1]{\texttt{#1}}% document class option name
\newenvironment{docspec}{\begin{quote}\noindent}{\end{quote}}% command specification environment

\newtheorem{theorem}{Theorem}
\newtheorem{corollary}{Corollary}
\newtheorem{definition}{Definition}
\newenvironment{justification} {\begin{proof}[Justification]} {\end{proof}}

\begin{document}

\maketitle % this prints the handout title, author, and date

\begin{abstract}
\noindent
We can apply the concept of \textbf{chemical potential} to describe and quantify the effect of a
solute on certain thermodynamic properties of a solution. These \textbf{colligative properties} 
include the lowering of vapour pressure of the solvent, the elevation of its boiling point, the 
depression of its freezing point, and the origin of osmotic pressure; they define a class
of real solutions that diverge from ideal solutions.
\end{abstract}

%\printclassoptions

In equilibrium systems, the chemical potential of a substance is the same in every phase in which it occurs.
(How do we know this is the case?)

\section*{Lowering of Vapour Pressure}

$$\Delta p_A = p_A - p^{*}_{A} = -x_B p^*_A$$

Where $x_B$ is the\dots

\section*{Boiling Point Elevation}

\section*{Freezing Point Depression}

\section*{Osmotic Pressure}

Here we derive a relation between the osmotic pressure and the molar concentration of solute.

\begin{equation}
    \Pi V = RT n_B
\end{equation}

This is the van 't Hoff equation for osmotic pressure. It bears striking resemblance to the ideal gas law. \marginnote{Not to be confused for the relation that describes the temperature dependence of the equilibrium constant, $K$.}
Replacing $\ce{[B]} = n_B / V$, we get the osmotic pressure relation:

\begin{equation}
    \Pi = \ce{[B]}RT
\end{equation}

This relation is valid only for ideal solutions.

\section{Electrochemistry Dump}
\textbf{Half-cell reactions.} Overall reaction can be conceived as two entirely separate reactions,
occurring at each electrode.

\begin{theorem}[Nernst Equation] Allows us to relate cell potential to the actual concentration of species
  present and the standard reduction potential ($E^\standardstate$).
  \begin{equation*}
    E = E^\standardstate - \frac{RT}{nF} \ln Q = E^\standardstate - \frac{RT}{nF} \ln\left(\frac{\alpha_B}{\alpha_A}\right)
\end{equation*}

\begin{proof}
  Suppose that the conventional cell reaction has the following general form
  \begin{equation*}
    \ce{aA + bB -> pP + qQ}.
  \end{equation*}

  $\Delta_{\mathrm{r}}G$ can be written in terms of individual chemical potentials.
  \begin{equation*}
    \Delta_{\mathrm{r}}G = p\mu_P + q\mu_Q - a\mu_A - b\mu_B
  \end{equation*}
  (of course, this is at standard state; above equation needs to change)
\end{proof}
  
\end{theorem}

\section{Summary}

\begin{itemize}
  \item The First Law relates heat, work and the change in internal energy.
  \item Heat and work are path functions, whereas internal energy is a state function.
  \item By convention the heat $q$ is positive when heat is absorbed by the system and $w$ is positive when work is done on the system.
\end{itemize}

\bibliography{lec23}
\bibliographystyle{plainnat}



\end{document}
