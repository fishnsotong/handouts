\section{Microscopy}

\subsection*{Optical Microscopy}

Lens equation. $$\frac{1}{u} + \frac{1}{v} = \frac{1}{f}$$

\textbf{Diffraction limit.} Resolution for optical microscopy is limited by \textit{diffraction}. This limit is ~250 nm for light microscopy 
and ~100 nm for confocal. See \textit{Abbe's theorem}. $$a = \frac{\lambda}{2n \sin \alpha}$$

Linear resolution limit.

Angular resolution limit.

Important thing is not to be confused between both.

\subsection*{Electron Microscopy}

Richardson's law. \textit{figure out what this is}

\begin{equation*}
    J_c = A_c T^2 \exp \left(- \frac{\Psi}{k_B T}\right)
\end{equation*}

Different varieties of electron microscopy exist for varying purposes.

Scanning electron microscopy (SEM).

Transmission electron microscopy (TEM).

\subsection*{Aberrations (not examined)}
Total optical aberration in a system takes into account the following sources of optical error with the equation below.
\begin{equation*}
    d_{\mathrm{tot}} = \sqrt{d_0^2 + d_{\mathrm{C}}^2 + d_{\mathrm{S}}^2 + d_{\mathrm{d}}^2}
\end{equation*}

Source size.
\begin{equation*}
    d_0 = \sqrt{\frac{4i}{\pi^2 \beta \alpha^2}}
\end{equation*}

Chromatic aberration.
\begin{equation*}
    d_{\mathrm{C}} = \left( \frac{\Delta E}{E_0}C_{\mathrm{C}} \alpha \right)
\end{equation*}

Spherical aberration.
\begin{equation*}
    d_{\mathrm{S}}  = \frac{1}{2} C_{\mathrm{S}} \alpha^3
\end{equation*}

\textbf{Diffraction limit.} Which is the angular resolution for EM.
\begin{equation*}
    d_{\mathrm{d}} = \frac{1.22 \lambda}{\alpha}
\end{equation*}