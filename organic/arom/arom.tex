\documentclass{article} % For LaTeX2e
\usepackage{amsmath,amsthm} 
\usepackage[euler]{textgreek} % For greek letters in normal text
\usepackage[version=4]{mhchem} % For typical chemical formulas and symbols

\usepackage{../../nips13submit_e,times}
\usepackage[hidelinks]{hyperref}
\usepackage{url}
%\documentstyle[nips13submit_09,times,art10]{article} % For LaTeX 2.09


\title{Heteroaromatics}


\author{
Wayne Yeo\thanks{wwzyeo@gmail.com} \\
Imperial College London
}

% The \author macro works with any number of authors. There are two commands
% used to separate the names and addresses of multiple authors: \And and \AND.
%
% Using \And between authors leaves it to \LaTeX{} to determine where to break
% the lines. Using \AND forces a linebreak at that point. So, if \LaTeX{}
% puts 3 of 4 authors names on the first line, and the last on the second
% line, try using \AND instead of \And before the third author name.

\newcommand{\fix}{\marginpar{FIX}}
\newcommand{\new}{\marginpar{NEW}}

\newtheorem{theorem}{Theorem}
\newtheorem{corollary}{Corollary}
\newenvironment{justification} {\begin{proof}[Justification]} {\end{proof}}

\theoremstyle{definition}
\newtheorem{definition}{Definition}
\newtheorem{example}{Example}


\nipsfinalcopy % Uncomment for camera-ready version

\begin{document}

\maketitle

Here we review the synthesis and reactivity of 

\section{Typical reactivity}

Hückel Molecular Orbital Theory predicts that

The chemistry of organometallic compounds is largely driven by the unique character
of the \ce{C-M} bond. Because metals are much more electropositive than carbon, these
bonds are characterised by their high degree of polarity, which leads to the \textsigma\space
bond being polarised towards carbon.

\begin{table}[ht]
   \caption{Comapring Pauling electronegativities}
   \label{electronegativity-table}
   \begin{center}
   \begin{tabular}{ll}
   \multicolumn{1}{c}{\bf Element}  &\multicolumn{1}{c}{\bf{Pauling scale} $\chi$}
   \\ \hline \\
   Li         &0.98 \\
   Mg         &1.31 \\
   Ce         &1.12 \\
   Cu         &1.90 \\
   Zn         &1.65 \\
   C          &2.55 \\

   \end{tabular}
   \end{center}
   \end{table}
   

\subsection*{Solvents}
The solvents used in the preparation and execution of polar organometallic reactivity are all ethers.
Typically, this is diethyl ether (\ce{Et2O}) or THF. Other solvents that are sometimes used include
dioxane and dimethoxyethane.

\begin{figure}[ht]
   \begin{center}
      \includegraphics{ios_scan_3410543749.pdf}
   \end{center}
\end{figure}

\subsection*{Metal-halogen exchange}

In the following sections, we discuss the main classes of polar organometallic reagents
we encounter in synthesis. \textit{It may be easier to treat Grignard reagents as the
standard, and compare other organometallics to these}.

\section{Grignard reagents}

\subsection{Use in 1,4 conjugate addition}

Typically, Grignard reagents have a high charge density and are \textbf{hard nucleophiles},
strongly favouring reactions with hard electrophilic centres according to HSAB theory. As such,
we would typically expect them to perform 1,2 direct addition to an \textalpha,\textbeta-unsaturated
carbonyl.

\section{Organolithiums}

\ce{C-M} bonds are polarised to an extremely high degree, more so than Grignard reagents.
In fact, MeLi, which is the simplest organolithium, is an ionic solid.

\subsection{Lithium-halogen exchange}

\textit{Why doesn't the organolithium decompose?} When \ce{CH3Li} or \textit{n}-\ce{BuLi} is
used in halogen-metal exchange, a rather eletrophilic \ce{MeX} or \textit{n}-\ce{BuX} is obtained
as a byproduct. The resulting alkyl halide can undergo S\textsubscript{N}2 substituion with the
organolithium compound as a nucleophile. Given that the organolithium species is highly reactive,
we would expect this side reaction to be a concern.

However, S\textsubscript{N}2 reactions of organolithium compounds with alkyl halides are rather
slow at the low temperatures at which halogen-metal exchange is usually executed, so this side
reaction is usually not a problem. This is a good example of \textbf{reaction kinetic control}, 
where there is insufficient kinetic energy to cross the activation barrier for thermodynamically
favourable side reactions.

\section{Organocuprates}

Organocuprates, or dialkylcopper reagents are far less reactive than their lithium or
magnesium counterparts, and can therefore be useful at introducing \textbf{chemoselectivity} in synthesis.

\section{Organozinc}

\section{Organocerium}

Though not a typical part of a typical undergraduate chemistry curriculum, organocerium 
reagents are interesting because they are so different from what we typically encounter.
These reagents also offer an import synthetic utility.





\section{General formatting instructions}
\label{gen_inst}

The text must be confined within a rectangle 5.5~inches (33~picas) wide and
9~inches (54~picas) long. The left margin is 1.5~inch (9~picas).
Use 10~point type with a vertical spacing of 11~points. Times New Roman is the
preferred typeface throughout. Paragraphs are separated by 1/2~line space,
with no indentation.

Paper title is 17~point, initial caps/lower case, bold, centered between
2~horizontal rules. Top rule is 4~points thick and bottom rule is 1~point
thick. Allow 1/4~inch space above and below title to rules. All pages should
start at 1~inch (6~picas) from the top of the page.

%The version of the paper submitted for review should have ``Anonymous Author(s)'' as the author of the paper.

For the final version, authors' names are
set in boldface, and each name is centered above the corresponding
address. The lead author's name is to be listed first (left-most), and
the co-authors' names (if different address) are set to follow. If
there is only one co-author, list both author and co-author side by side.

Please pay special attention to the instructions in section \ref{others}
regarding figures, tables, acknowledgments, and references.

\section{Headings: first level}
\label{headings}

First level headings are lower case (except for first word and proper nouns),
flush left, bold and in point size 12. One line space before the first level
heading and 1/2~line space after the first level heading.

\subsection{Headings: second level}

Second level headings are lower case (except for first word and proper nouns),
flush left, bold and in point size 10. One line space before the second level
heading and 1/2~line space after the second level heading.

\subsubsection{Headings: third level}

Third level headings are lower case (except for first word and proper nouns),
flush left, bold and in point size 10. One line space before the third level
heading and 1/2~line space after the third level heading.

\section{Citations, figures, tables, references}
\label{others}

These instructions apply to everyone, regardless of the formatter being used.

\subsection{Citations within the text}

Citations within the text should be numbered consecutively. The corresponding
number is to appear enclosed in square brackets, such as [1] or [2]-[5]. The
corresponding references are to be listed in the same order at the end of the
paper, in the \textbf{References} section. (Note: the standard
\textsc{Bib\TeX} style \texttt{unsrt} produces this.) As to the format of the
references themselves, any style is acceptable as long as it is used
consistently.

As submission is double blind, refer to your own published work in the 
third person. That is, use ``In the previous work of Jones et al.\ [4]'',
not ``In our previous work [4]''. If you cite your other papers that
are not widely available (e.g.\ a journal paper under review), use
anonymous author names in the citation, e.g.\ an author of the
form ``A.\ Anonymous''. 


\subsection{Footnotes}

Indicate footnotes with a number\footnote{Sample of the first footnote} in the
text. Place the footnotes at the bottom of the page on which they appear.
Precede the footnote with a horizontal rule of 2~inches
(12~picas).\footnote{Sample of the second footnote}

\subsection{Figures}

All artwork must be neat, clean, and legible. Lines should be dark
enough for purposes of reproduction; art work should not be
hand-drawn. The figure number and caption always appear after the
figure. Place one line space before the figure caption, and one line
space after the figure. The figure caption is lower case (except for
first word and proper nouns); figures are numbered consecutively.

Make sure the figure caption does not get separated from the figure.
Leave sufficient space to avoid splitting the figure and figure caption.

You may use color figures. 
However, it is best for the
figure captions and the paper body to make sense if the paper is printed
either in black/white or in color.
\begin{figure}[h]
\begin{center}
%\framebox[4.0in]{$\;$}
\fbox{\rule[-.5cm]{0cm}{4cm} \rule[-.5cm]{4cm}{0cm}}
\end{center}
\caption{Sample figure caption.}
\end{figure}

\subsection{Tables}

All tables must be centered, neat, clean and legible. Do not use hand-drawn
tables. The table number and title always appear before the table. See
Table~\ref{sample-table}.

Place one line space before the table title, one line space after the table
title, and one line space after the table. The table title must be lower case
(except for first word and proper nouns); tables are numbered consecutively.

\begin{table}[t]
\caption{Sample table title}
\label{sample-table}
\begin{center}
\begin{tabular}{ll}
\multicolumn{1}{c}{\bf PART}  &\multicolumn{1}{c}{\bf DESCRIPTION}
\\ \hline \\
Dendrite         &Input terminal \\
Axon             &Output terminal \\
Soma             &Cell body (contains cell nucleus) \\
\end{tabular}
\end{center}
\end{table}

\section{Final instructions}
Do not change any aspects of the formatting parameters in the style files.
In particular, do not modify the width or length of the rectangle the text
should fit into, and do not change font sizes (except perhaps in the
\textbf{References} section; see below). Please note that pages should be
numbered.

\subsubsection*{Recommended reading}

A good textbook on calculus ar the level required by the physical sciences
is the book, \emph{Calculus with Analytic Geometry} by George Simmons. For
a more formal treatment, see books both named \emph{Calculus} by Tom Apostol
and Michael Spivak.

\subsubsection*{References}

References follow the acknowledgments. Use unnumbered third level heading for
the references. Any choice of citation style is acceptable as long as you are
consistent. It is permissible to reduce the font size to `small' (9-point) 
when listing the references. {\bf Remember that this year you can use
a ninth page as long as it contains \emph{only} cited references.}

\pagebreak

\small{
[1] Alexander, J.A. \& Mozer, M.C. (1995) Template-based algorithms
for connectionist rule extraction. In G. Tesauro, D. S. Touretzky
and T.K. Leen (eds.), {\it Advances in Neural Information Processing
Systems 7}, pp. 609-616. Cambridge, MA: MIT Press.

[2] Bower, J.M. \& Beeman, D. (1995) {\it The Book of GENESIS: Exploring
Realistic Neural Models with the GEneral NEural SImulation System.}
New York: TELOS/Springer-Verlag.

[3] Hasselmo, M.E., Schnell, E. \& Barkai, E. (1995) Dynamics of learning
and recall at excitatory recurrent synapses and cholinergic modulation
in rat hippocampal region CA3. {\it Journal of Neuroscience}
{\bf 15}(7):5249-5262.
}

\end{document}